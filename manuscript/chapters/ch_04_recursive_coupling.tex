\chapter{Recursive Coupling and Depth Fields}

\section{Overview}

Self-reference is fundamental to meaning. The act of thinking about thinking, or using language to describe language, creates recursive loops that both stabilize and transform semantic structures. These feedback mechanisms are formalized through recursive coupling, creating a analyzable structure \autocite{Barabasi2016}. It also provides a basis for understanding hetero-recursive phenomena like metaphor and analogy, in which concepts from one semantic domain get mapped onto another. This chapter introduces the tensors that govern these processes. Their interplay generates forces which shape the manifold and give rise to complexity and emergent patterns of thought.

\section{Recursive Coupling Tensor \(R_{ijk}(p, q, t)\)}

The recursive coupling tensor captures nonlocal, bidirectional influences through a mixed partial derivative formulation. It is analogous to a second-order field interaction and formalizes the interdependence of recursive effects across the manifold:
\begin{equation}
R_{ijk}(p, q, t) = \frac{\partial^2 C_k(p,t)}{\partial \psi_i(p) \partial \psi_j(q)}
\end{equation}

where \(\psi_i(p)\) denotes the \(i\)-th component of the semantic field at point \(p\), and \(C_k(p,t)\) is the \(k\)-th component of the coherence field at \(p\) and time \(t\). This encodes how recursive activity at point \(q\) modulates the coherence structure at point \(p\) through cross-sensitivity in the semantic field.

\section{Dual Character of the Recursive Coupling Tensor}

The recursive coupling tensor \(R_{ijk}(p, q, t)\) exhibits a dual mathematical character requiring careful treatment. This duality reflects fundamental tension between operational definition and dynamical evolution in field theories dealing with recursive systems.

The tensor simultaneously serves two distinct mathematical roles:
\begin{enumerate}
    \item Operational Definition: As a second derivative of the coherence field,
    \begin{equation}
    R_{ijk}(p, q, t) = \frac{\partial^2 C_k(p, t)}{\partial \psi_i(p) \partial \psi_j(q)}
    \end{equation}
    \item Dynamical Evolution: As an independent field evolving according to
    \begin{equation}
    \frac{dR_{ijk}(p, q, t)}{dt} = \Phi(C_{\mathrm{mag}}(p, t)) \cdot \chi_{ijk}(p, q, t)
    \end{equation}
\end{enumerate}

For mathematical coherence, these two perspectives must align through a compatibility condition:
\begin{equation}
\frac{d}{dt}\left(\frac{\partial^2 C_k(p, t)}{\partial \psi_i(p) \partial \psi_j(q)}\right) = \Phi(C_{\mathrm{mag}}(p, t)) \cdot \chi_{ijk}(p, q, t)
\end{equation}

This requirement places nontrivial constraints on the dynamics of underlying semantic fields \(\psi_i\). It may require additional terms in the evolution equations for \(\psi_i\). The constraint likely depends on a separation of timescales between rapid field adjustments and slower structural evolution. The precise analytic mechanism by which this compatibility is realized represents an active area of theoretical development in RFT.

\section{Recursive Depth \(D(p, t)\)}

The recursive depth field \(D(p, t)\) is a scalar function that specifies the maximal recursion depth sustainable at point \(p\) before coherence falls below a threshold:
\begin{equation}
D(p, t) = \max \left\{ d \in \mathbb{N} : \left| \frac{\partial^d C(p,t)}{\partial \psi^d} \right| \geq \epsilon \right\}
\end{equation}

where \(\epsilon\) is the minimum coherence signal threshold.

Interpretation:
\begin{itemize}
    \item Concepts with low \(D\) (e.g., elementary arithmetic) exhibit shallow recursive structure.
    \item Structures with high \(D\) (e.g., persistent personal narratives or worldviews) maintain coherence across multiple recursive layers.
\end{itemize}

\section{Recursive Stress-Energy Tensor \(T_{ij}^{\text{rec}}\)}

The recursive stress-energy tensor \(T_{ij}^{\text{rec}}\) characterizes how recursion induces deformation within the semantic manifold. It describes the coupling between recursive dynamics and semantic curvature, analogous to the stress-energy tensor in general relativity.
\begin{equation}
T_{ij}^{\text{rec}} = \rho(p,t) \cdot v_i(p,t) v_j(p,t) + P_{ij}(p,t)
\end{equation}

where
\begin{itemize}
    \item \(\rho(p,t) = \frac{1}{\det(g_{ij})}\) is the constraint density, with higher values corresponding to regions of greater local semantic mass,
    \item \(v_i(p,t) = \frac{d}{dt} \psi_i(p,t)\) is the velocity of recursive change in the \(i\)-th component of the semantic field,
    \item \(P_{ij}(p,t)\) is the recursive pressure tensor, defined as
\end{itemize}
\begin{equation}
P_{ij} = \gamma(\nabla_i v_j + \nabla_j v_i) + \eta g_{ij} \nabla_k v^k
\end{equation}

with
\begin{itemize}
    \item \(\gamma\) denoting the elasticity of recursive loops,
    \item \(\eta\) representing resistance to bulk recursive collapse,
    \item \(\nabla_i\) the covariant derivative with respect to the manifold's geometry.
\end{itemize}

\section{Hetero-Recursive Coupling and Cross-Domain Mapping}

The recursive coupling tensor \(R_{ijk}(p, q, t)\) operates within and across semantic subdomains, making it possible to formalize metaphor, analogy, and cross-modal recursion.

\subsection{Domain Structure in Semantic Space}

The semantic manifold \(\mathcal{M}\) is partitioned into a collection of submanifolds (domains):
\begin{equation}
\mathcal{M} = \bigcup_{d=1}^{N_D} \mathcal{M}_d
\end{equation}

where
\begin{itemize}
    \item \(\mathcal{M}_d\) denotes a semantic domain with its own intrinsic metric \(g_{ij}^{(d)}\),
    \item Domains are connected via interface regions equipped with transition functions.
\end{itemize}

Examples include linguistic, visual, embodied, logical, emotional, and narrative spaces, each characterized by distinct semantic organization.

\subsection{Self vs. Hetero-Recursive Coupling}

The recursive coupling tensor decomposes as
\begin{equation}
R_{ijk}(p, q, t) = R_{ijk}^{\text{self}}(p, q, t) + R_{ijk}^{\text{hetero}}(p, q, t)
\end{equation}

where
\begin{itemize}
    \item \(R_{ijk}^{\text{self}}(p, q, t) = R_{ijk}(p, q, t) \cdot \delta_{d(p),d(q)}\) corresponds to intra-domain (self-referential) recursion,
    \item \(R_{ijk}^{\text{hetero}}(p, q, t) = R_{ijk}(p, q, t) \cdot (1 - \delta_{d(p),d(q)})\) corresponds to inter-domain (hetero-referential) recursion,
    \item \(d(p)\) returns the domain index of \(p\),
    \item \(\delta_{d(p),d(q)}\) is the Kronecker delta.
\end{itemize}

This decomposition separates recursive feedback within a domain from cross-domain recursive mapping.

\subsection{Cross-Domain Mapping Formalism}

Hetero-recursive coupling requires explicit mechanisms to map between distinct semantic spaces. A domain translation tensor addresses this:
\begin{equation}
T_{ij}^{(d \to d')} : T\mathcal{M}_d \to T\mathcal{M}_{d'}
\end{equation}

which maps tangent spaces between domains, allowing coherence in one domain to influence another even when their organizational principles differ.

The cross-domain recursive coupling is then given by
\begin{equation}
R_{ijk}^{\text{hetero}}(p, q, t) = \chi_{ijl}(p, q, t) \cdot T_{lk}^{(d(q) \to d(p))}
\end{equation}

where
\begin{itemize}
    \item \(\chi_{ijl}(p, q, t)\) is the latent recursive channel tensor encoding potential connectivity,
    \item \(T_{lk}^{(d(q) \to d(p))}\) translates recursive influence from domain \(d(q)\) to domain \(d(p)\).
\end{itemize}

\subsection[The Role of chi_ijk in Cross-Domain Mapping]{The Role of \(\chi_{ijk}\) in Cross-Domain Mapping}

The latent recursive channel tensor \(\chi_{ijk}(p, q, t)\) forms the substrate for cross-domain recursion, encoding:
\begin{enumerate}
    \item Potential connectivity between semantic regions, irrespective of domain,
    \item Channel capacity for recursive flow between points,
    \item Similarity structure that governs analogical mapping.
\end{enumerate}

Its evolution is described by
\begin{equation}
\frac{d\chi_{ijk}(p, q, t)}{dt} = \alpha \cdot S_{ij}(p, q) \cdot N_k + \beta \cdot H(p, q, t) \cdot G_{ijk} - \gamma \cdot D_{ijk}(p, q)
\end{equation}

where
\begin{itemize}
    \item \(S_{ij}(p, q)\) is the rank-2 semantic similarity tensor,
    \item \(N_k\) is a basis vector in the \(k\)-dimension, promoting \(S_{ij}\) to rank-3,
    \item \(H(p, q, t)\) is the scalar historical co-activation strength,
    \item \(G_{ijk}\) is a rank-3 geometric structure tensor distributing \(H\) across dimensions,
    \item \(D_{ijk}(p, q)\) is the rank-3 domain incompatibility tensor.
\end{itemize}

These terms maintain tensor rank consistency and shape the evolution of \(\chi_{ijk}\) appropriately.

\section{Metaphor and Analogy as Hetero-Recursive Structures}

Metaphors and analogies are formalized as stable hetero-recursive mappings between domains. A metaphor \(\mathcal{M}\) from source domain \(\mathcal{S}\) to target domain \(\mathcal{T}\) is defined as
\begin{equation}
\mathcal{M}_{\mathcal{S} \to \mathcal{T}} = \{(p, q, R_{ijk}^{\text{hetero}}(p, q, t)) \mid p \in \mathcal{S},\ q \in \mathcal{T},\ \|R_{ijk}^{\text{hetero}}(p, q, t)\| > \epsilon\}
\end{equation}

The stability of the metaphoric structure is quantified by
\begin{equation}
\text{Stab}(\mathcal{M}_{\mathcal{S} \to \mathcal{T}}) = \frac{\int_{\mathcal{S} \times \mathcal{T}} \|R_{ijk}^{\text{hetero}}(p, q, t)\| \cdot W(p, t) \cdot W(q, t) \, dp \, dq}{\int_{\mathcal{S} \times \mathcal{T}} \|R_{ijk}^{\text{hetero}}(p, q, t)\| \, dp \, dq}
\end{equation}

where \(W(p, t)\) and \(W(q, t)\) are weighting functions. High-stability metaphoric mappings persist and exert significant influence on the organization of cognitive structures across domains. These correspond to "conceptual metaphors" in cognitive linguistics.

\subsection{Cross-Domain Recursive Amplification}

When hetero-recursive coupling forms closed loops across domains, cross-amplification of coherence may arise:
\begin{equation}
C_i^{(d)}(p, t+1) = f\left(C_i^{(d)}(p, t), \sum_{d' \neq d} \int_{\mathcal{M}_{d'}} R_{ijk}^{\text{hetero}}(p, q, t) \cdot C_j^{(d')}(q, t) \, dq\right)
\end{equation}

Such feedback circuits stabilize cross-domain mappings and can result in:
\begin{enumerate}
    \item Metaphoric entrenchment: mappings that become automatic within the cognitive architecture,
    \item Conceptual blending: the emergence of hybrid domains at the interface of recursive loops,
    \item Semantic innovation: the formation of novel conceptual structures from previously unconnected domains.
\end{enumerate}