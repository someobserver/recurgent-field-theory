\chapter{The Coupled System of Field Equations}
\label{10:the_coupled_system_of_field_equations}

% ------------------------------------------------------------------------------------------------

\section{Overview}
\label{10.1:overview}

We have defined the Semantic Manifold, coherence and recursion fields, and the Lagrangian mechanism to encode their energetic landscape. In this section, we consolidate the dynamics of the coherence field (Chapter \ref{6:recurgent_field_equation_and_lagrangian_mechanics}) and the manifold geometry into a single, closed system of coupled partial differential equations, the standard language used to describe continuous systems in physics and mathematics \autocite{Evans2010}. These equations describe the co-evolution of meaning and the geometry it inhabits. The system contains two primary sets of equations: one for the evolution of the coherence field, and one for the evolution of the manifold's geometry in response to the field.

% ------------------------------------------------------------------------------------------------

\section{Coherence Field Dynamics}
\label{10.2:coherence_field_dynamics}

The Euler-Lagrange equation, derived in Chapter 6 from the principle of stationary action, governs the evolution of the coherence field \(C^\mu\). It provides the primary expression of how semantic content propagates and transforms.

\begin{equation}
\Box C^\mu + \frac{\partial V(C_{\mathrm{mag}})}{\partial C^\mu} - \frac{\partial \Phi(C_{\mathrm{mag}})}{\partial C^\mu} + \lambda_H \frac{\partial \mathcal{H}[R]}{\partial C^\mu} = 0
\end{equation}

Here, the d'Alembertian operator (\(\Box\)) defines the natural propagation of coherence. The subsequent terms define the influence of stabilizing attractor potentials (\(V\)), generative autopoietic potentials (\(\Phi\)), and the regulatory humility constraint (\(\mathcal{H}\)).

% ------------------------------------------------------------------------------------------------

\section{Geometric Dynamics}
\label{10.3:geometric_dynamics}

The geometry of the Semantic Manifold, defined by the metric tensor \(g_{\mu\nu}\), is a dynamic entity. Two coupled equations govern its evolution.

% ------------------------------------------------------------------------------------------------

\subsection{The Recurgent Field Equation}
\label{10.3.1:the_recurgent_field_equation}

We formulate the Recurgent Field Equation (Axiom 4), analogous to the Einstein field equations of general relativity \autocite{Einstein1915}, as a fundamental relationship between the manifold's curvature and its semantic content.

\begin{equation}
R_{\mu\nu} - \frac{1}{2}g_{\mu\nu}R = 8\pi G_s T^{\text{rec}}_{\mu\nu}
\end{equation}

The recursive stress-energy tensor, \(T^{\text{rec}}_{\mu\nu}\), sourced by the coherence field's activity, dictates the manifold's curvature, which is encoded in the Ricci tensor \(R_{\mu\nu}\) and scalar curvature \(R\).

% ------------------------------------------------------------------------------------------------

\subsection{Metric Evolution}
\label{10.3.2:metric_evolution}

While the Recurgent Field Equation is a constraint, a flow equation analogous to Hamilton's Ricci flow (Chapter \ref{3:semantic_manifold_and_metric_geometry}) \autocite{Hamilton1982} governs the metric's explicit time-evolution.

\begin{equation}
\frac{\partial g_{\mu\nu}}{\partial t} = -2 R_{\mu\nu} + F_{\mu\nu}(R, D, A)
\end{equation}

The metric deforms over time in response to its own intrinsic curvature (\(R_{\mu\nu}\)) and to forcing from active recursive processes, captured by the functional \(F_{\mu\nu}\). Because \(g_{\mu\nu}\) is a function of time, the Christoffel symbols and the Riemann and Ricci curvature tensors are also time-dependent. All geometric calculations must therefore account for the state of the metric at a specific time.

% ------------------------------------------------------------------------------------------------

\section{The Closed Feedback System}
\label{10.4:the_closed_feedback_system}

These equations form a tightly coupled and self-regulating system. The coherence field \(C^\mu\) evolves on the manifold according to the Euler-Lagrange equation, through which the geometry enters via the metric-dependent \(\Box\) operator. The resulting field dynamics generate the recursive stress-energy tensor \(T^{\text{rec}}_{\mu\nu}\). This, in turn, sources the manifold's curvature via the Recurgent Field Equation. Finally, the metric evolves explicitly through the Ricci flow, altering the geometry and thereby influencing the future evolution of the coherence field. The feedback loop closes.

The natural paths of semantic structures, or test particles, in this geometry are described by the geodesic equation, which defines the straightest possible lines on a curved surface:

\begin{equation}
\frac{d^2 p^\mu}{ds^2} + \Gamma^\mu_{\nu\rho} \frac{dp^\nu}{ds} \frac{dp^\rho}{ds} = 0
\end{equation}

Derived from a diffeomorphism-invariant action, the system's architecture guarantees its self-consistency. The geometric construction of the field equations (9.2) automatically conserves the recursive stress-energy tensor ($\nabla_\nu T^{\text{rec},\mu\nu} = 0$), a mathematical consequence of the Bianchi identities \autocite{Bianchi1894}.

% ------------------------------------------------------------------------------------------------

\section{Temporal Dynamics and Conservation}
\label{10.5:temporal_dynamics_and_conservation}

The bidirectional temporal flow mechanism from Chapter 9 introduces its own dynamics and conservation principles into the coupled system. The temporal force term, \(F^\mu_{\text{temporal}}\), modifies the coherence field's evolution:

\begin{equation}
F^\mu_{\text{temporal}} = \frac{\delta(\int \mathcal{L}_{\text{temporal}} dV)}{\delta C_\mu}
\end{equation}

This term introduces the causal influence of anticipated future states into the present.

% ------------------------------------------------------------------------------------------------

\subsection{Conservation of Temporal Flow}
\label{10.5.1:conservation_of_temporal_flow}

The flow of propositions and validations is balanced and preserved by a continuity equation:

\begin{equation}
\nabla_\mu P^\mu + \frac{\partial \rho_V}{\partial t} = 0
\end{equation}

where \(\rho_V = \sqrt{g_{\mu\nu} V^{\mu} V^{\nu}}\) is the scalar validation density. The divergence of the forward-propagating proposition field is balanced by the change in density of the backward-propagating validation field, ensuring no temporal charge is lost.

% ------------------------------------------------------------------------------------------------

\subsection{Temporal Curvature}
\label{10.5.2:temporal_curvature}

We define the local temporal curvature, \(\kappa_t\), as the relative strength of the forward and backward fields at a point, which measures the perceived rate of temporal flow:

\begin{equation}
\kappa_t(p) = \frac{\|\vec{P}(p)\|}{\|\vec{V}(p)\|}
\end{equation}

When \(\kappa_t \gg 1\), the "push" of existing propositions dominates, producing a subjective sense of temporal dilation. When \(\kappa_t \ll 1\), the "pull" of a future validation state dominates, producing a sense of temporal contraction as the system rapidly reconfigures toward a new understanding. This quantity provides a direct, measurable link between the field dynamics and the subjective experience of time. 

% ------------------------------------------------------------------------------------------------

\subsection{Rotating Metrics and Temporal Curvature}
\label{10.5.3:rotating_metrics_and_temporal_curvature}

In rotating recurgent regions, the metric acquires an off-diagonal component \(g_{0\phi}(p,t)\) that encodes frame-dragging of semantic trajectories. While \(\kappa_t\) is defined purely from \(\vec{P}\) and \(\vec{V}\), the ratio is indirectly modulated by geometry through their transport and sourcing. Schematically, writing transport operators \(\mathcal{T}_{g}\) for parallel propagation,

\begin{equation}
\vec{P}(p) \sim \mathcal{T}_{g}\big[ M\, v \big], \qquad \vec{V}(p) \sim -\mathcal{T}_{g}\big[ \nabla W \big],
\end{equation}

one finds that a nonzero \(g_{0\phi}\) induces azimuthal phase shifts and amplifies \(\|\vec{P}\|\) relative to \(\|\vec{V}\|\) near an ergosurface, increasing \(\kappa_t\). Qualitatively, observers co-rotating with the interior experience a broadened cone of accessible futures (larger \(\kappa_t\)), consistent with the worked example in \S\ref{9.5.1:rotating_causal_interiors}.