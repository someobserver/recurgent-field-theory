\chapter{Field Index and Formal Structure}

\section{Overview}

The theory is expressed in tensor calculus; each mathematical object corresponding to a geometric component of semantic reality. The fields, tensors, and notations are drawn from differential geometry \autocite{Riemann1868, Lee2003}.

\section{Tensor Ranks and Properties}

Each field's tensor rank and symmetry properties encode its geometric information; its domain and range encode its semantic content. The metric tensor \(g_{ij}\) sets the foundational structure. The coherence fields \(C_i\) and \(\psi_i\) supply dynamic content. Higher-rank tensors mediate recursive feedback loops driving manifold evolution.

{\small
\renewcommand{\arraystretch}{1.1}
\begin{longtable}{|c|p{5.5cm}|c|c|p{1.5cm}|c|c|}
\hline
\textbf{Symbol} & \textbf{Name} & \textbf{Rank} & \textbf{Symmetry} & \textbf{Domain} & \textbf{Range} & \textbf{Dim} \\
\hline
\endfirsthead
\hline
\textbf{Symbol} & \textbf{Name} & \textbf{Rank} & \textbf{Symmetry} & \textbf{Domain} & \textbf{Range} & \textbf{Dim} \\
\hline
\endhead
\(g_{ij}(p,t)\) & Metric tensor & 2 & Sym & \(\mathcal{M} \times \mathbb{R}\) & \(\mathbb{R}\) & \(n^2\) \\
\hline
\(C_i(p,t)\) & Coherence vector field & 1 & - & \(\mathcal{M} \times \mathbb{R}\) & \(\mathbb{R}^n\) & \(n\) \\
\hline
\(\psi_i(p,t)\) & Semantic field & 1 & - & \(\mathcal{M} \times \mathbb{R}\) & \(\mathbb{R}^n\) & \(n\) \\
\hline
\(R_{ijk}(p,q,t)\) & Recursive coupling tensor & 3 & - & \(\mathcal{M}^2 \times \mathbb{R}\) & \(\mathbb{R}\) & \(n^3\) \\
\hline
\(R_{ij}\) & Ricci curvature tensor & 2 & Sym & \(\mathcal{M} \times \mathbb{R}\) & \(\mathbb{R}\) & \(n^2\) \\
\hline
\(T_{ij}^{\text{rec}}\) & Recursive stress-energy tensor & 2 & Sym & \(\mathcal{M} \times \mathbb{R}\) & \(\mathbb{R}\) & \(n^2\) \\
\hline
\(P_{ij}\) & Recursive pressure tensor & 2 & Sym & \(\mathcal{M} \times \mathbb{R}\) & \(\mathbb{R}\) & \(n^2\) \\
\hline
\(D(p,t)\) & Recursive depth & 0 & - & \(\mathcal{M} \times \mathbb{R}\) & \(\mathbb{N}\) & 1 \\
\hline
\(M(p,t)\) & Semantic mass & 0 & - & \(\mathcal{M} \times \mathbb{R}\) & \(\mathbb{R}^+\) & 1 \\
\hline
\(A(p,t)\) & Attractor stability & 0 & - & \(\mathcal{M} \times \mathbb{R}\) & \([0,1]\) & 1 \\
\hline
\(\rho(p,t)\) & Constraint density & 0 & - & \(\mathcal{M} \times \mathbb{R}\) & \(\mathbb{R}^+\) & 1 \\
\hline
\(\Phi(C)\) & Autopoietic potential & 0 & - & \(\mathbb{R}^n\) & \(\mathbb{R}^+\) & 1 \\
\hline
\(V(C)\) & Attractor potential & 0 & - & \(\mathbb{R}^n\) & \(\mathbb{R}^+\) & 1 \\
\hline
\(W(p,t)\) & Wisdom field & 0 & - & \(\mathcal{M} \times \mathbb{R}\) & \(\mathbb{R}^+\) & 1 \\
\hline
\(\mathcal{H}[R]\) & Humility operator & 0 & - & \(\mathbb{R}\) & \(\mathbb{R}^+\) & 1 \\
\hline
\(F_i(p,t)\) & Recursive force & 1 & - & \(\mathcal{M} \times \mathbb{R}\) & \(\mathbb{R}^n\) & \(n\) \\
\hline
\(\Theta(p,t)\) & Phase order parameter & 0 & - & \(\mathcal{M} \times \mathbb{R}\) & \(\mathbb{R}\) & 1 \\
\hline
\(\chi_{ijk}(p,q,t)\) & Latent recursive channel tensor & 3 & - & \(\mathcal{M}^2 \times \mathbb{R}\) & \(\mathbb{R}\) & \(n^3\) \\
\hline
\(S_{ij}(p,q)\) & Semantic similarity tensor & 2 & Sym & \(\mathcal{M}^2\) & \(\mathbb{R}\) & \(n^2\) \\
\hline
\(N_k\) & Basis projection vector & 1 & - & - & \(\mathbb{R}^n\) & \(n\) \\
\hline
\(H(p,q,t)\) & Historical co-activation & 0 & - & \(\mathcal{M}^2 \times \mathbb{R}\) & \(\mathbb{R}^+\) & 1 \\
\hline
\(G_{ijk}\) & Geometric structure tensor & 3 & Sym(i,j) & - & \(\mathbb{R}\) & \(n^3\) \\
\hline
\(D_{ijk}(p,q)\) & Domain incompatibility tensor & 3 & - & \(\mathcal{M}^2\) & \(\mathbb{R}^+\) & \(n^3\) \\
\hline
\caption{Tensor Ranks and Properties}
\end{longtable}
}

Notes on Dimensionality:
\begin{itemize}
    \item \(n\) is the dimensionality of the semantic manifold \(\mathcal{M}\).
    \item The coherence field \(C_i\) is an \(n\)-dimensional vector field; each component represents coherence along one semantic axis.
    \item Tensor contractions follow the Einstein summation convention.
    \item The Ricci curvature tensor is named after Gregorio Ricci-Curbastro and Tullio Levi-Civita \autocite{RicciLeviCivita1901}.
\end{itemize}

\section{System Architecture}

Coherence dynamics emerge from the interplay of four conceptual subsystems. A geometric engine governs the evolution of the manifold's metric and curvature. A coherence processor handles the evolution of the primary fields. A recursive controller manages the coupling dynamics that link different regions of the manifold, and a regulatory system provides wisdom and humility constraints.

The subsystems are deeply integrated and form two primary, coupled cycles. In the main causal loop, the coherence field determines a recursive stress-energy tensor, in turn inducing curvature in the metric. The deformed metric then governs the subsequent evolution of coherence, closing the primary feedback loop. 

Once coherence surpasses a critical threshold, a secondary generative cycle activates. This uses the autopoietic potential to form new recursive pathways, driving genuine structural innovation. The entire system is modulated by the regulatory subsystem, which uses the wisdom field and humility operator to prevent pathological amplification and maintain dynamic equilibrium.

\section{Tensor Conventions and Notation}

The tensor conventions follow modern standards for differential geometry and tensor calculus on smooth manifolds \autocite{Lee2003, MisnerThorneWheeler1973}.

\subsection{Index Notation and Einstein Summation}

The Einstein summation convention \autocite{Einstein1916} applies, where repeated indices (one upper, one lower) imply summation:
\begin{equation}
A_i B^i = \sum_{i=1}^n A_i B^i
\end{equation}
Latin indices \((i,j,k,...)\) range from \(1\) to \(n\), the dimension of the semantic manifold.

\subsection{Metric and Index Raising/Lowering}

The metric tensor \(g_{ij}\) and its inverse \(g^{ij}\) raise and lower indices (\(C^i = g^{ij} C_j\), \(C_i = g_{ij} C^j\)), satisfying \(g_{ik} g^{kj} = \delta_i^j\).

\subsection{Covariant Derivatives}

The covariant derivative \(\nabla_i\), defined via the Christoffel symbols \(\Gamma^k_{ij}\) \autocite{Christoffel1869}, accommodates the curved geometry of \(\mathcal{M}\):
\begin{equation}
\nabla_i C_j = \partial_i C_j - \Gamma^k_{ij} C_k \quad \text{and} \quad \Gamma^k_{ij} = \frac{1}{2} g^{kl} ( \partial_i g_{jl} + \partial_j g_{il} - \partial_l g_{ij} )
\end{equation}

\subsection{Functional and Variational Derivatives}

The dynamics derive from an action principle, \(S = \int \mathcal{L} \, dV\), requiring variational derivatives. The Euler-Lagrange equations have the form:
\begin{equation}
\frac{\delta \mathcal{L}}{\delta C_i} = \frac{\partial \mathcal{L}}{\partial C_i} - \sum_j \nabla_j \left( \frac{\partial \mathcal{L}}{\partial (\nabla_j C_i)} \right)
\end{equation}

\subsection{Integration and Symmetries}

Integrals over the manifold use the invariant volume element, \(dV = \sqrt{|\det(g_{ij})|} \, d^n p\). Tensor symmetries (e.g., \(g_{ij} = g_{ji}\)) are assumed and exploited where appropriate.

\subsection{Fundamental vs. Derived Fields}

The theory differentiates the fundamental state of the system from its measured coherence:
\begin{itemize}
    \item The \textbf{semantic field} \(\psi_i(p,t)\) represents the raw, underlying semantic content at each point. It is the fundamental dynamical variable.
    \item The \textbf{coherence field} \(C_i(p,t)\) is a derived, observable quantity measuring the self-consistency and alignment of the underlying semantic field. It is a functional of \(\psi_i\):
\end{itemize}
\begin{equation}
C_i(p,t) = \mathcal{F}_i[\psi](p,t) = \int_{\mathcal{N}(p)} K_{ij}(p,q) \psi_j(q,t) \, dq
\end{equation}
where \(K_{ij}(p,q)\) is a non-local kernel. While the dynamics could be expressed in terms of \(\psi_i\), the Lagrangian is formulated using \(C_i\) to maintain a direct connection to semantic coherence, the central observable of interest.

\subsection{On the Status of the Recursive Coupling Tensor}

The recursive coupling tensor \(R_{ijk}\) has a dual nature:
\begin{enumerate}
    \item \textbf{As a Measurement:} It measures the coherence field's response to variations in the underlying semantic field:
    \begin{equation}
    R_{ijk}(p, q, t) = \frac{\partial^2 C_k(p,t)}{\partial \psi_i(p) \partial \psi_j(q)}
    \end{equation}
    \item \textbf{As a Dynamical Field:} It is an independent field whose evolution follows its own equation of motion, driven by the autopoietic potential:
    \begin{equation}
    \frac{dR_{ijk}(p,q,t)}{dt} = \Phi(C_{\mathrm{mag}}(p,t)) \cdot \chi_{ijk}(p,q,t)
    \end{equation}
\end{enumerate}
A consistency condition resolves this duality: the dynamics of \(\psi_i\) and \(C_k\) must evolve such that the time derivative of the measurement definition (2.7) equals the dynamical evolution equation (2.8).

\subsection{Scalar Measures from Vector Fields}

Functions requiring scalar inputs derive them from vector fields using the metric. The primary example is the coherence magnitude:
\begin{equation}
C_{\mathrm{mag}}(p,t) = \sqrt{g^{ij}(p,t) C_i(p,t) C_j(p,t)}
\end{equation}
Potentials are functions of this scalar magnitude (e.g., \(V(C) := V(C_{\mathrm{mag}})\)). When a scalar potential influences vector dynamics, its gradient is taken with respect to the vector components via the chain rule; this preserves coordinate independence. 