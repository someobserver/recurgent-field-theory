\chapter{Field Index and Formal Structure}

\section{Overview}

The theory is expressed in tensor calculus each mathematical object in correspondence with a geometric component of semantic reality, drawing from the work of Riemann \autocite{Riemann1868}. This section inventories the fields and tensors used throughout the following chapters.

\section{Tensor Ranks and Properties}

Each field in RFT carries geometric information through its tensor rank and symmetry properties. The fields also carry semantic content through their domain and range specifications. The metric tensor \(g_{ij}\) quantifies the foundational structure for this. Coherence fields \(C_i\) and \(\psi_i\) provide the dynamic content which drives manifold evolution. Higher-rank tensors like \(R_{ijk}\) mediate feedback loops.

The semantic manifold evolves through the fields it supports. This evolution requires careful attention to how tensorial structures couple and transform.

{\footnotesize
\begin{longtable}{|p{2.5cm}|p{4cm}|c|c|p{2.5cm}|c|c|}
\hline
\textbf{Symbol} & \textbf{Name} & \textbf{Rank} & \textbf{Symmetry} & \textbf{Domain} & \textbf{Range} & \textbf{Dim} \\
\hline
\endfirsthead
\hline
\textbf{Symbol} & \textbf{Name} & \textbf{Rank} & \textbf{Symmetry} & \textbf{Domain} & \textbf{Range} & \textbf{Dim} \\
\hline
\endhead
\hline
\(g_{ij}(p,t)\) & Metric tensor & 2 & Sym & \(\mathcal{M} \times \mathbb{R}\) & \(\mathbb{R}\) & \(n^2\) \\
\hline
\(C_i(p,t)\) & Coherence vector field & 1 & - & \(\mathcal{M} \times \mathbb{R}\) & \(\mathbb{R}^n\) & \(n\) \\
\hline
\(\psi_i(p,t)\) & Semantic field & 1 & - & \(\mathcal{M} \times \mathbb{R}\) & \(\mathbb{R}^n\) & \(n\) \\
\hline
\(R_{ijk}(p,q,t)\) & Recursive coupling tensor & 3 & - & \(\mathcal{M}^2 \times \mathbb{R}\) & \(\mathbb{R}\) & \(n^3\) \\
\hline
\(R_{ij}\) & Ricci curvature tensor \autocite{RicciLeviCivita1901} & 2 & Sym & \(\mathcal{M} \times \mathbb{R}\) & \(\mathbb{R}\) & \(n^2\) \\
\hline
\(T_{ij}^{\text{rec}}\) & Recursive stress-energy tensor & 2 & Sym & \(\mathcal{M} \times \mathbb{R}\) & \(\mathbb{R}\) & \(n^2\) \\
\hline
\(P_{ij}\) & Recursive pressure tensor & 2 & Sym & \(\mathcal{M} \times \mathbb{R}\) & \(\mathbb{R}\) & \(n^2\) \\
\hline
\(D(p,t)\) & Recursive depth & 0 & - & \(\mathcal{M} \times \mathbb{R}\) & \(\mathbb{N}\) & 1 \\
\hline
\(M(p,t)\) & Semantic mass & 0 & - & \(\mathcal{M} \times \mathbb{R}\) & \(\mathbb{R}^+\) & 1 \\
\hline
\(A(p,t)\) & Attractor stability & 0 & - & \(\mathcal{M} \times \mathbb{R}\) & \([0,1]\) & 1 \\
\hline
\(\rho(p,t)\) & Constraint density & 0 & - & \(\mathcal{M} \times \mathbb{R}\) & \(\mathbb{R}^+\) & 1 \\
\hline
\(\Phi(C)\) & Autopoietic potential & 0 & - & \(\mathbb{R}^n\) & \(\mathbb{R}^+\) & 1 \\
\hline
\(V(C)\) & Attractor potential & 0 & - & \(\mathbb{R}^n\) & \(\mathbb{R}^+\) & 1 \\
\hline
\(W(p,t)\) & Wisdom field & 0 & - & \(\mathcal{M} \times \mathbb{R}\) & \(\mathbb{R}^+\) & 1 \\
\hline
\(\mathcal{H}[R]\) & Humility operator & 0 & - & \(\mathbb{R}\) & \(\mathbb{R}^+\) & 1 \\
\hline
\(F_i(p,t)\) & Recursive force & 1 & - & \(\mathcal{M} \times \mathbb{R}\) & \(\mathbb{R}^n\) & \(n\) \\
\hline
\(\Theta(p,t)\) & Phase order parameter & 0 & - & \(\mathcal{M} \times \mathbb{R}\) & \(\mathbb{R}\) & 1 \\
\hline
\(\chi_{ijk}(p,q,t)\) & Latent recursive channel tensor & 3 & - & \(\mathcal{M}^2 \times \mathbb{R}\) & \(\mathbb{R}\) & \(n^3\) \\
\hline
\(S_{ij}(p,q)\) & Semantic similarity tensor & 2 & Sym & \(\mathcal{M}^2\) & \(\mathbb{R}\) & \(n^2\) \\
\hline
\(N_k\) & Basis projection vector & 1 & - & - & \(\mathbb{R}^n\) & \(n\) \\
\hline
\(H(p,q,t)\) & Historical co-activation & 0 & - & \(\mathcal{M}^2 \times \mathbb{R}\) & \(\mathbb{R}^+\) & 1 \\
\hline
\(G_{ijk}\) & Geometric structure tensor & 3 & Sym(i,j) & - & \(\mathbb{R}\) & \(n^3\) \\
\hline
\(D_{ijk}(p,q)\) & Domain incompatibility tensor & 3 & - & \(\mathcal{M}^2\) & \(\mathbb{R}^+\) & \(n^3\) \\
\hline
\caption{Tensor Ranks and Properties}
\end{longtable}
}

Notes on Dimensionality:
\begin{itemize}
    \item \(n\) is the dimensionality of the semantic manifold \(\mathcal{M}\)
    \item The coherence field \(C_i\) is an \(n\)-dimensional vector field, each component representing coherence along one semantic axis
    \item Tensor contractions (e.g., \(g^{ij}(\nabla_i C_k)(\nabla_j C^k)\)) follow standard Einstein summation convention
\end{itemize}

\section{Coupled Field Equations}

The primary interdependencies between fields form a closed loop of recursive influence:

Semantic mass curves metric space. \rightarrow Curved space shapes coherence flow. \rightarrow Coherence flow generates recursive coupling. \rightarrow Recursive coupling reshapes the metric.

These equations formalize the closed loop:

Coherence Evolution:
\begin{equation}
\Box C_i = T^{\text{rec}}_{ij} \cdot g^{jk} C_k
\end{equation}

Metric Evolution:
\begin{equation}
\frac{\partial g_{ij}}{\partial t} = -2 R_{ij} + F_{ij}(R, D, A)
\end{equation}

Recursive Coupling Evolution:
\begin{equation}
\frac{dR_{ijk}(p,q,t)}{dt} = \Phi(C(p,t)) \cdot \chi_{ijk}(p,q,t)
\end{equation}

Semantic Mass Composition:
\begin{equation}
M(p,t) = D(p,t) \cdot \rho(p,t) \cdot A(p,t)
\end{equation}

Wisdom Dynamics:
\begin{equation}
\frac{dW}{dt} = \alpha C \cdot \frac{d(\nabla_f R)}{dt} + \beta \nabla_f R \cdot \frac{dC}{dt} + \gamma C \cdot \nabla_f R \cdot \frac{dP}{dt}
\end{equation}

Where scalar measures are used for consistency:
\begin{itemize}
    \item \(C\) refers to the scalar magnitude \(C_{\mathrm{mag}} = \sqrt{g^{ij}C_i C_j}\)
    \item \(\nabla_f R\) refers to the scalar magnitude of the forecast gradient
    \item \(P\) refers to the scalar magnitude of the pressure tensor \(P_{mag} = \sqrt{g^{ij}g^{kl}P_{ik}P_{jl}}\)
\end{itemize}

The field equations create interdependent relationships through mathematical coupling. The dependency structure follows from the axioms:

\subsection{System Architecture and Mathematical Dependencies}

Field dynamics unfold in interconnected processes organized into four subsystems: (1) a geometric engine governing metric and curvature operations, (2) a coherence processor managing field evolution, (3) a recursive controller regulating coupling dynamics, and (4) a regulatory system enforcing wisdom and constraint mechanisms.

The system architecture has two coupled cycles regulated by a wisdom-humility cascade. The primary causal loop establishes geometric-semantic coupling through coherence field evolution. The resulting coherence field $C$ encodes local semantic consistency at each manifold point, determining the recursive stress-energy tensor $T^{\text{rec}}$, which quantifies semantic pressure from coherence. That tensor induces curvature via the Ricci tensor $R_{ij}$, deforming the metric $g_{ij}$ analogous to mass-energy effects in general relativity. The deformed metric modulates coherence gradients $\nabla C$, establishing principal directions for semantic propagation and governing the subsequent evolution of $C$, completing the causal loop.

When the coherence field $C$ surpasses critical thresholds, a generative cycle activates via autopoietic potential $\Phi(C)$. The system's capacity for structural innovation produces the recursive coupling tensor $R_{ijk}$, encoding formation of new recursive pathways to reinforce and stabilize the coherence field. The coherence field simultaneously defines an attractor potential $V$ corresponding to stable semantic basins. The interplay between the autopoietic potential $\Phi(C)$ and attractor potential $V(C)$ determines system stability.

The regulatory subsystem prevents pathological amplification with wisdom and humility mechanisms. The recursive coupling tensor $R_{ijk}$ determines the forecast gradient $\nabla_f R$, encoding system sensitivity to anticipated future states. The resulting gradient underpins the wisdom field $W$, representing adaptive, foresight-weighted coherence to modulate the humility operator $H$. Humility functions as a regulatory damping factor on recursive amplification, constraining semantic mass $M$ to limit excessive or unstable recurgent growth.

Semantic mass emerges through compositional relations involving recursive depth $D$ (maximal recursion layers sustaining coherence), constraint density $\rho$ (derived from the metric tensor determinant), and attractor stability $A$ (resistance to perturbation). The magnitude of semantic mass $M = D \cdot \rho \cdot A$ determines the influence of semantic structures on their local environment. Resulting gravitational-like effects govern subsequent evolution of the semantic field.

The metric tensor $g_{ij}$ determines constraint density $\rho$, where higher constraint corresponds to denser semantic packing. The recursive pressure tensor $P_{ij}$ modulates attractor stability $A$, supporting persistence of stable structures. The velocity field $v_i$ governs pressure generation $P_{ij}$, with the rate of semantic change directly influencing local pressure dynamics.

Stable semantic structures emerge from the dynamic equilibrium between generative recursion and constraint geometry. Emergent, inherent regulatory mechanisms prevent runaway or pathological recurgent configurations.

\section{Tensor Conventions and Notation}

The tensor conventions used throughout this framework are explicitly defined, following the modern standards for differential geometry and tensor calculus on smooth manifolds \autocite{Lee2003}.

\subsection{Index Notation and Einstein Summation}

Adopting the Einstein summation convention \autocite{Einstein1916}, where repeated indices (one upper, one lower) imply summation:
\begin{equation}
A_i B^i = \sum_{i=1}^n A_i B^i
\end{equation}

Indices follow these conventions:
\begin{itemize}
    \item Latin indices \((i,j,k,...)\) range from \(1\) to \(n\), where \(n\) is the dimension of the semantic manifold
    \item Greek indices \((\mu,\nu,\alpha,...)\) are used when working in local coordinate systems or parameter spaces
    \item Repeated indices appearing in upper and lower positions indicate summation
    \item Free indices must match on both sides of any equation
\end{itemize}

\subsection{Metric and Index Raising/Lowering}

The metric tensor \(g_{ij}(p,t)\) and its inverse \(g^{ij}(p,t)\) are used consistently to raise and lower indices:
\begin{equation}
C^i = g^{ij} C_j
\end{equation}
\begin{equation}
C_i = g_{ij} C^j
\end{equation}

The metric satisfies:
\begin{equation}
g_{ik} g^{kj} = \delta_i^j
\end{equation}

Where \(\delta_i^j\) is the Kronecker delta. This relationship holds at each point \(p\) and time \(t\), even as the metric evolves.

\subsection{Covariant Derivatives}

The covariant derivative \(\nabla_i\) accounts for the curved geometry of the semantic manifold:
\begin{equation}
\nabla_i C_j = \partial_i C_j - \Gamma^k_{ij} C_k
\end{equation}
\begin{equation}
\nabla_i C^j = \partial_i C^j + \Gamma^j_{ik} C^k
\end{equation}

Where \(\Gamma^k_{ij}\) are the Christoffel symbols \autocite{Christoffel1869}:
\begin{equation}
\Gamma^k_{ij} = \frac{1}{2} g^{kl} \left( \partial_i g_{jl} + \partial_j g_{il} - \partial_l g_{ij} \right)
\end{equation}

Covariant derivatives keep the tensor equations coordinate-independent across the curved semantic manifold.

\subsection{Functional Derivatives}

When working with the Lagrangian and action principles, functional derivatives are used, defined as:
\begin{equation}
\frac{\delta \mathcal{L}}{\delta C_i(p)} = \lim_{\epsilon \to 0} \frac{\mathcal{L}[C_i + \epsilon \delta_p C_i] - \mathcal{L}[C_i]}{\epsilon}
\end{equation}

Where \(\delta_p C_i\) represents a variation localized at point \(p\). This differs from the partial derivative \(\frac{\partial \mathcal{L}}{\partial C_i}\), which applies to the Lagrangian density as a function rather than a functional.

In discrete implementations, the functional derivative becomes:
\begin{equation}
\frac{\delta \mathcal{L}}{\delta C_i(p)} \approx \frac{\partial \mathcal{L}}{\partial C_i(p)} - \sum_j \nabla_j \left( \frac{\partial \mathcal{L}}{\partial (\nabla_j C_i(p))} \right)
\end{equation}

This formulation accounts for both local and gradient terms in the Lagrangian.

\subsection{Tensor Symmetries}

When tensors possess symmetries, they are explicitly noted:
\begin{itemize}
    \item Symmetric tensors: \(T_{ij} = T_{ji}\) (e.g., the metric tensor \(g_{ij}\))
    \item Antisymmetric tensors: \(A_{ij} = -A_{ji}\)
    \item Partially symmetric tensors: Symmetry only in specific index groups
\end{itemize}

These symmetries constrain the independent components and affect how contractions and operations are performed.

\subsection{Integration Measures}

Integrals over the semantic manifold incorporate the metric-dependent volume element:
\begin{equation}
\int_{\mathcal{M}} f(p) \, dV_p = \int_{\mathcal{M}} f(p) \sqrt{|\det(g_{ij})|} \, d^n p
\end{equation}

This preserves coordinate independence of integrated quantities and reflects the curved geometry of semantic space.

\subsection{Tensor Density Weights}

Some quantities (like the constraint density \(\rho\)) behave as tensor densities rather than pure tensors:
\begin{equation}
\rho(p,t) = \frac{1}{\det(g_{ij})}
\end{equation}

When integrating such densities, appropriate transformation rules maintain coordinate invariance.

\subsection{Fundamental and Derived Field Relationships}

For theoretical consistency, the relationship between fundamental and derived fields requires explicit definition:

Semantic Field vs. Coherence Field:
\begin{itemize}
    \item The semantic field \(\psi_i(p,t)\) represents the fundamental state variables of the system, or raw semantic content at each point
    \item The coherence field \(C_i(p,t)\) is a derived field that measures the self-consistency of semantic patterns:
\end{itemize}
\begin{equation}
C_i(p,t) = \mathcal{F}_i[\psi](p,t) = \int_{\mathcal{N}(p)} K_{ij}(p,q) \psi_j(q,t) \, dq
\end{equation}

Where:
\begin{itemize}
    \item \(\mathcal{F}_i\) is the coherence functional operator
    \item \(K_{ij}(p,q)\) is a non-local kernel measuring semantic alignment between points \(p\) and \(q\)
    \item \(\mathcal{N}(p)\) is a neighborhood around point \(p\)
\end{itemize}

This relationship allows derivatives of \(C\) to be expressed with respect to \(\psi\):
\begin{equation}
\frac{\partial C_k(p,t)}{\partial \psi_i(q)} = K_{ki}(p,q)
\end{equation}

And second derivatives as used in the recursive coupling tensor:
\begin{equation}
\frac{\partial^2 C_k(p,t)}{\partial \psi_i(p') \partial \psi_j(q')} = \frac{\partial K_{ki}(p,p')}{\partial \psi_j(q')}
\end{equation}

While the action principle could be formulated directly in terms of \(\psi_i\), using \(C_i\) as the primary dynamical variable provides a more direct connection to semantic coherence, the central observable of interest. The Lagrangian is thus expressed in terms of \(C_i\) with the understanding that it is functionally dependent on the underlying semantic field \(\psi_i\).

For computational implementations, the distinction between \(\psi_i\) and \(C_i\) becomes particularly important when:
\begin{enumerate}
    \item Initializing field configurations
    \item Interpreting field evolution
    \item Calculating recursive properties that depend on derivatives with respect to \(\psi_i\)
\end{enumerate}

In simulation contexts, both fields are typically tracked simultaneously, with \(\psi_i\) evolving according to its own dynamics and \(C_i\) updated according to the functional relationship above.

\subsection{Vector Fields and Derived Scalar Measures}

To maintain consistent tensor properties throughout RFT, vector fields must be properly converted when contexts require scalar values:

Coherence Field Scalar Measures:
The coherence field \(C_i(p,t)\) is a vector field (rank-1 tensor), but several functions require scalar measures derived from it:
\begin{equation}
C_{\mathrm{mag}}(p,t) = \sqrt{g^{ij}(p,t) C_i(p,t) C_j(p,t)}
\end{equation}

This scalar magnitude measure quantifies the total coherence strength independent of direction. A normalized coherence projection may be defined:
\begin{equation}
C_{proj}(p,t) = \frac{C_i(p,t) \cdot v^i(p,t)}{|v(p,t)|}
\end{equation}

Where \(v^i(p,t)\) is a local reference direction (often the semantic velocity field).

Usage in Scalar Functions and Thresholds:
All potential functions and thresholds use these scalar measures rather than the vector field directly:
\begin{itemize}
    \item Attractor potential: \(V(C) := V(C_{\mathrm{mag}})\)
    \item Autopoietic potential: \(\Phi(C) := \Phi(C_{\mathrm{mag}})\)
    \item Thresholds: \(C_{\mathrm{mag}} > C_{threshold}\)
\end{itemize}

Scalar-to-Vector Influences:
When scalar functions influence vector dynamics, the effect is distributed using tensor promotion mechanisms:
\begin{equation}
\frac{\partial \Phi(C_{\mathrm{mag}})}{\partial C_i} = \frac{\partial \Phi}{\partial C_{\mathrm{mag}}} \cdot \frac{\partial C_{\mathrm{mag}}}{\partial C_i} = \frac{\partial \Phi}{\partial C_{\mathrm{mag}}} \cdot \frac{g^{ij}C_j}{C_{\mathrm{mag}}}
\end{equation}

Gradients of scalar potentials shape vector field dynamics independent of coordinate choice.

All equations in RFT should be interpreted with this convention unless explicitly stated otherwise.

\subsection{Status of Recursive Coupling Tensor \(R_{ijk}\)}

The recursive coupling tensor \(R_{ijk}(p,q,t)\) requires precise characterization for mathematical consistency:

Hybrid Field Status:
\(R_{ijk}\) has a dual nature:
\begin{enumerate}
    \item Measurement Interpretation: The expression in Section 2.1
    \begin{equation}
    R_{ijk}(p, q, t) = \frac{\partial^2 C_k(p,t)}{\partial \psi_i(p) \partial \psi_j(q)}
    \end{equation}
    provides a measurement interpretation or operational definition of \(R_{ijk}\). That is, how recursive coupling can be detected and measured through its effects on the coherence field.
    \item Independent Dynamical Field: For the purposes of time evolution, \(R_{ijk}\) is treated as an independent field governed by:
    \begin{equation}
    \frac{dR_{ijk}(p,q,t)}{dt} = \Phi(C_{\mathrm{mag}}(p,t)) \cdot \chi_{ijk}(p,q,t)
    \end{equation}
\end{enumerate}

Resolution of Apparent Contradiction:
This dual perspective is reconciled by imposing a consistency requirement:
\begin{equation}
\frac{d}{dt}\left(\frac{\partial^2 C_k(p,t)}{\partial \psi_i(p) \partial \psi_j(q)}\right) = \Phi(C_{\mathrm{mag}}(p,t)) \cdot \chi_{ijk}(p,q,t)
\end{equation}

The dynamics of \(C_k\) and \(\psi_i\) satisfy this constraint. In practice, the evolution of \(\psi_i\) includes terms that maintain this relationship. Consistency is achieved through the coupled field system rather than by treating \(R_{ijk}\) as strictly derived.

Lagrangian Treatment:
In the Lagrangian formulation, \(R_{ijk}\) appears directly only through the humility operator \(\mathcal{H}[R]\). Variation of the action with respect to \(C_i\) incorporates the chain-rule effect through \(\psi_i\), which suffices to capture the coupling relationship. This avoids the need to vary \(R_{ijk}\) independently while preserving the physical interpretation of recursive coupling. 