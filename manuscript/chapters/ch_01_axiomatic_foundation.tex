\chapter{Axiomatic Foundation}
\label{1:axiomatic_foundation}

We found Recurgent Field Theory on axioms defining the geometric and dynamic properties of meaning. These posit a Semantic Manifold, a fundamental field of coherence, and recursive coupling principles governing their interaction. Such a pursuit follows from the foundational work put forth by Francis Crick and Christof Koch, hypothesizing that the phenomena of mind are accessible to the language of physics \autocite{Crick1990, KochConsciousness2019}.

% ------------------------------------------------------------------------------------------------

\section{Axiom 1: Semantic Manifold}
\label{1.1:axiom_1_semantic_manifold}

We propose that a differentiable manifold \(\mathcal{M}\) (semantic space), equipped with a dynamic metric tensor \(g_{\mu\nu}(p,t)\), defines the geometric structure of meaning. This work builds on the theory of conceptual spaces \autocite{Gardenfors2000}, which posits that meaning can be represented as a geometric structure.

\begin{equation}
g_{\mu\nu}(p,t) : \mathcal{M} \times \mathbb{R} \rightarrow \mathbb{R}
\end{equation}

\begin{equation}
ds^2 = g_{\mu\nu}(p,t) \, dp^\mu \, dp^\nu
\end{equation}

The structure of the Semantic Manifold defines distances, curvature, and geodesics in meaning-space, consistent with Riemannian geometry \autocite{Riemann1868}.

% ------------------------------------------------------------------------------------------------

\section{Axiom 2: Fundamental Semantic Field}
\label{1.2:axiom_2_fundamental_semantic_field}

We assert that a vector field \(\psi^\mu(p,t)\) on \(\mathcal{M}\) represents the semantic configuration. Coherence \(C^\mu(p,t)\) is a functional of \(\psi^\mu\). The concept of a field of forces operating in a psychological or semantic space has historical precedent in the "lifespace" or psychological field proposed by Kurt Lewin in his work in social psychology \autocite{Lewin1951}.

\begin{equation}
C^\mu(p,t) = \mathcal{F}^\mu[\psi](p,t)
\end{equation}

\begin{equation}
C_{\text{mag}}(p,t) = \sqrt{g_{\mu\nu}(p,t) C^\mu(p,t) C^\nu(p,t)}
\end{equation}

% ------------------------------------------------------------------------------------------------

\section{Axiom 3: Recursive Coupling}
\label{1.3:axiom_3_recursive_coupling}

We posit that a rank-3 tensor \(R^\rho_{\mu\nu}(p,q,t)\) quantifies the influence of activity at point \(q\) on coherence at point \(p\) through self-referential processes. This is a formalization of self-reference, which has long been understood as an emergence mechanism of complex meaning. The recursive coupling tensor provides a mathematical structure for the "strange loops" and "tangled hierarchies" that allow formal systems to achieve self-awareness and generate profound degrees of meaning \autocite{Hofstadter1979}.

\begin{equation}
R^\rho_{\mu\nu}(p,q,t) = \frac{\mathcal{D}^2 C^\rho(p,t)}{\mathcal{D} \psi^\mu(p) \mathcal{D} \psi^\nu(q)}
\end{equation}

% ------------------------------------------------------------------------------------------------

\section{Axiom 4: Geometric Coupling Principle}
\label{1.4:axiom_4_geometric_coupling_principle}

Semantic mass \(M(p,t)\) curves the manifold's geometry according to:

\begin{equation}
R_{\mu\nu} - \frac{1}{2}g_{\mu\nu}R = 8\pi G_s T^{\text{rec}}_{\mu\nu}
\end{equation}

We find that a Semantic Mass Equation is structurally analogous to the field equations of general relativity \autocite{Einstein1915, MisnerThorneWheeler1973, Wald1984}, where the recursive stress-energy tensor \(T^{\text{rec}}_{\mu\nu}\) is an analog of the mass-energy tensor in spacetime curvature. Here we define semantic mass as:

\begin{equation}
M(p,t) = D(p,t) \cdot \rho(p,t) \cdot A(p,t)
\end{equation}

\begin{equation}
\rho(p,t) = \frac{1}{\det(g_{\mu\nu}(p,t))}
\end{equation}

% ------------------------------------------------------------------------------------------------

\section{Axiom 5: Variational Evolution}
\label{1.5:axiom_5_variational_evolution}

We derive the dynamics of semantic fields from the principle of stationary action applied to the Lagrangian, where field dynamics preserve symmetries and conservation laws, consistent with the variational principle \autocite{GoldsteinPooleSafko2002, Arnold1989}.

\begin{equation}
\mathcal{L} = \frac{1}{2} g_{\mu\rho} g_{\nu\sigma} (\nabla^\rho C^\mu)(\nabla^\sigma C^\nu) - V(C_{\text{mag}}) + \Phi(C_{\text{mag}}) - \lambda_H \mathcal{H}[R]
\end{equation}

where

\begin{equation}
\frac{\delta S}{\delta C^\mu} = 0 \quad \text{and} \quad S = \int_{\mathcal{M}} \mathcal{L} \, dV
\end{equation}

% ------------------------------------------------------------------------------------------------

\section{Axiom 6: Autopoietic Threshold}
\label{1.6:axiom_6_autopoietic_threshold}

We recognize that when coherence magnitude exceeds a critical threshold, the autopoietic potential \(\Phi(C_{\text{mag}})\) becomes positive and drives generative phase transitions. This threshold marks the point at which a system achieves a state of self-producing and self-maintaining autonomy, first defined by Humberto Maturana and Francisco J. Varela in their seminal work on theoretical biology \autocite{MaturanaVarela1980}.

The transition to this state is a physical phenomenon of self-organization common to complex systems. We derive the mathematical language for such phase transitions from the field of synergetics \autocite{Haken1983}, where macroscopic order emerges from the collective behavior of microscopic components. Furthermore, the emergence of such order is an expected property of sufficiently complex networks, which naturally exhibit self-organizing criticality \autocite{BakTangWiesenfeld1987}.

\begin{equation}
\Phi(C_{\text{mag}}) = \begin{cases}
\alpha_{\Phi} (C_{\text{mag}} - C_{\text{threshold}})^{\beta_{\Phi}} & \text{if } C_{\text{mag}} \geq C_{\text{threshold}} \\
0 & \text{otherwise}
\end{cases}
\end{equation}

% ------------------------------------------------------------------------------------------------

\section{Axiom 7: Recurgence}
\label{1.7:axiom_7_recurgence}

We define a semantic system to possess the capacity for \textit{recurgency} if it can autoreferentially model and reconfigure its own semantic structure through geometric evolution. This capacity is formally characterized by the non-zero second-order evolution of the metric tensor:

\begin{equation}
\frac{\partial^2 g_{\mu\nu}}{\partial t^2} \neq 0
\end{equation} 