\chapter{Axiomatic Foundation}

Recurgent Field Theory is founded on axioms that define the geometric and dynamic properties of meaning. These axioms posit a semantic manifold, a fundamental field of coherence, and recursive coupling principles governing their interaction.

\section{Axiom 1: Semantic Manifold}

A differentiable manifold \(\mathcal{M}\) (semantic space), equipped with a dynamic metric tensor \(g_{ij}(p,t)\), defines the geometric structure of meaning.

\begin{equation}
g_{ij}(p,t) : \mathcal{M} \times \mathbb{R} \rightarrow \mathbb{R}
\end{equation}
\begin{equation}
ds^2 = g_{ij}(p,t) \, dp^i \, dp^j
\end{equation}

The structure of the semantic manifold defines distances, curvature, and geodesics in meaning-space, consistent with Riemannian geometry \autocite{Riemann1868}.

\section{Axiom 2: Fundamental Semantic Field}

A vector field \(\psi_i(p,t)\) on \(\mathcal{M}\) represents the semantic configuration. Coherence \(C_i(p,t)\) is a functional of \(\psi_i\).

\begin{equation}
C_i(p,t) = \mathcal{F}_i[\psi](p,t)
\end{equation}
\begin{equation}
C_{\text{mag}}(p,t) = \sqrt{g^{ij}(p,t) C_i(p,t) C_j(p,t)}
\end{equation}

\section{Axiom 3: Recursive Coupling}

A rank-3 tensor \(R_{ijk}(p,q,t)\) quantifies the influence of activity at point \(q\) on coherence at point \(p\) through self-referential processes.

\begin{equation}
R_{ijk}(p,q,t) = \frac{\partial^2 C_k(p,t)}{\partial \psi_i(p) \partial \psi_j(q)}
\end{equation}

\section{Axiom 4: Geometric Coupling Principle}

Semantic mass \(M(p,t)\) curves the manifold geometry according to:

\begin{equation}
R_{ij} - \frac{1}{2}g_{ij}R = 8\pi G_s T^{\text{rec}}_{ij}
\end{equation}

The semantic mass equation is structurally analogous to the field equations of general relativity \autocite{Einstein1915, MisnerThorneWheeler1973, Wald1984}. The recursive stress-energy tensor \(T^{\text{rec}}_{ij}\) is analogous to the mass-energy tensor in spacetime curvature.

Here

\begin{equation}
M(p,t) = D(p,t) \cdot \rho(p,t) \cdot A(p,t)
\end{equation}
\begin{equation}
\rho(p,t) = \frac{1}{\det(g_{ij}(p,t))}
\end{equation}

\section{Axiom 5: Variational Evolution}

The dynamics of semantic fields derive from the principle of stationary action applied to the Lagrangian:

\begin{equation}
\mathcal{L} = \frac{1}{2} g^{ij} (\nabla_i C_k)(\nabla_j C^k) - V(C_{\text{mag}}) + \Phi(C_{\text{mag}}) - \lambda \mathcal{H}[R]
\end{equation}

where

\begin{equation}
\frac{\delta S}{\delta C_i} = 0 \quad \text{and} \quad S = \int_{\mathcal{M}} \mathcal{L} \, dV
\end{equation}

Semantic field dynamics preserve symmetries and conservation laws, consistent with the variational principle \autocite{GoldsteinPooleSafko2002, Arnold1989}.

\section{Axiom 6: Autopoietic Threshold}

When coherence magnitude exceeds a critical threshold, the autopoietic potential \(\Phi(C_{\text{mag}})\) becomes positive and drives generative phase transitions:

\begin{equation}
\Phi(C_{\text{mag}}) = \begin{cases}
\alpha (C_{\text{mag}} - C_{\text{threshold}})^{\beta} & \text{if } C_{\text{mag}} \geq C_{\text{threshold}} \\
0 & \text{otherwise}
\end{cases}
\end{equation} 