\chapter{Axiomatic Foundation}
\label{1:axiomatic_foundation}

We found Recurgent Field Theory on axioms defining the geometric and dynamic properties of meaning. These posit a Semantic Manifold, a fundamental field of coherence, and recursive coupling principles governing their interaction.

% ------------------------------------------------------------------------------------------------

\section{Axiom 1: Semantic Manifold}
\label{1.1:axiom_1_semantic_manifold}

We propose that a differentiable Manifold \(\mathcal{M}\) (semantic space), equipped with a dynamic metric tensor \(g_{ij}(p,t)\), defines the geometric structure of meaning. This work builds on the theory of conceptual spaces \autocite{Gardenfors2000}, which posits that meaning can be represented as a geometric structure.

\begin{equation}
g_{ij}(p,t) : \mathcal{M} \times \mathbb{R} \rightarrow \mathbb{R}
\end{equation}

\begin{equation}
ds^2 = g_{ij}(p,t) \, dp^i \, dp^j
\end{equation}

The structure of the Semantic Manifold defines distances, curvature, and geodesics in meaning-space, consistent with Riemannian geometry \autocite{Riemann1868}.

% ------------------------------------------------------------------------------------------------

\section{Axiom 2: Fundamental Semantic Field}
\label{1.2:axiom_2_fundamental_semantic_field}

We assert that a vector field \(\psi_i(p,t)\) on \(\mathcal{M}\) represents the semantic configuration. Coherence \(C_i(p,t)\) is a functional of \(\psi_i\). The concept of a field of forces operating in a psychological or semantic space has historical precedent in the "lifespace" or psychological field proposed by Kurt Lewin in his work in social psychology \autocite{Lewin1951}.

\begin{equation}
C_i(p,t) = \mathcal{F}_i[\psi](p,t)
\end{equation}

\begin{equation}
C_{\text{mag}}(p,t) = \sqrt{g^{ij}(p,t) C_i(p,t) C_j(p,t)}
\end{equation}

% ------------------------------------------------------------------------------------------------

\section{Axiom 3: Recursive Coupling}
\label{1.3:axiom_3_recursive_coupling}

We posit that a rank-3 tensor \(R_{ijk}(p,q,t)\) quantifies the influence of activity at point \(q\) on coherence at point \(p\) through self-referential processes. This is a formalization of self-reference, which has long been understood as an emergence mechanism of complex meaning. The recursive coupling tensor provides a mathematical structure for the "strange loops" and "tangled hierarchies" that allow formal systems to achieve self-awareness and generate profound degrees of meaning \autocite{Hofstadter1979}.

\begin{equation}
R_{ijk}(p,q,t) = \frac{\partial^2 C_k(p,t)}{\partial \psi_i(p) \partial \psi_j(q)}
\end{equation}

% ------------------------------------------------------------------------------------------------

\section{Axiom 4: Geometric Coupling Principle}
\label{1.4:axiom_4_geometric_coupling_principle}

Semantic mass \(M(p,t)\) curves the manifold geometry according to:

\begin{equation}
R_{ij} - \frac{1}{2}g_{ij}R = 8\pi G_s T^{\text{rec}}_{ij}
\end{equation}

We find that the Semantic Mass Equation is structurally analogous to the field equations of general relativity \autocite{Einstein1915, MisnerThorneWheeler1973, Wald1984}, where the recursive stress-energy tensor \(T^{\text{rec}}_{ij}\) is an analog of the mass-energy tensor in spacetime curvature. Here we define semantic mass as:

\begin{equation}
M(p,t) = D(p,t) \cdot \rho(p,t) \cdot A(p,t)
\end{equation}

\begin{equation}
\rho(p,t) = \frac{1}{\det(g_{ij}(p,t))}
\end{equation}

% ------------------------------------------------------------------------------------------------

\section{Axiom 5: Variational Evolution}
\label{1.5:axiom_5_variational_evolution}

We derive the dynamics of semantic fields from the principle of stationary action applied to the Lagrangian, where field dynamics preserve symmetries and conservation laws, consistent with the variational principle \autocite{GoldsteinPooleSafko2002, Arnold1989}.

\begin{equation}
\mathcal{L} = \frac{1}{2} g^{ij} (\nabla_i C_k)(\nabla_j C^k) - V(C_{\text{mag}}) + \Phi(C_{\text{mag}}) - \lambda \mathcal{H}[R]
\end{equation}

where

\begin{equation}
\frac{\delta S}{\delta C_i} = 0 \quad \text{and} \quad S = \int_{\mathcal{M}} \mathcal{L} \, dV
\end{equation}

% ------------------------------------------------------------------------------------------------

\section{Axiom 6: Autopoietic Threshold}
\label{1.6:axiom_6_autopoietic_threshold}

We recognize that when coherence magnitude exceeds a critical threshold, the autopoietic potential \(\Phi(C_{\text{mag}})\) becomes positive and drives generative phase transitions. This threshold marks the point at which a system achieves a state of self-producing and self-maintaining autonomy, first defined by Humberto Maturana and Francisco J. Varela in their seminal work on theoretical biology \autocite{MaturanaVarela1980}.

The transition to this state is a physical phenomenon of self-organization common to complex systems. We derive the mathematical language for such phase transitions from the field of synergetics \autocite{Haken1983}, where macroscopic order emerges from the collective behavior of microscopic components. Furthermore, the emergence of such order is an expected property of sufficiently complex networks, which naturally exhibit self-organizing criticality \autocite{BakTangWiesenfeld1987}.

\begin{equation}
\Phi(C_{\text{mag}}) = \begin{cases}
\alpha (C_{\text{mag}} - C_{\text{threshold}})^{\beta} & \text{if } C_{\text{mag}} \geq C_{\text{threshold}} \\
0 & \text{otherwise}
\end{cases}
\end{equation}

% ------------------------------------------------------------------------------------------------

\section{Axiom 7: Recurgence}
\label{1.7:axiom_7_recurgence}

We define a semantic system to possess the capacity for \textit{recurgency} if it can autoreferentially model and reconfigure its own semantic structure through geometric evolution. This capacity is formally characterized by the non-zero second-order evolution of the metric tensor:

\begin{equation}
\frac{\partial^2 g_{ij}}{\partial t^2} \neq 0
\end{equation} 