\chapter{Axiomatic Foundation}
\label{1:axiomatic_foundation}

We found Recurgent Field Theory on seven axioms defining geometric and dynamic properties of meaning. They posit a Semantic Manifold and a fundamental field of coherence, as well as recursive coupling principles governing their interaction. Such a pursuit grounds itself in Galileo's foundational insight that natural phenomena yield to mathematical description \autocite{Galilei1623}. It extends through the work of Francis Crick and Christof Koch, who hypothesize that the phenomena of mind are likewise accessible to the language of physics \autocite{Crick1990, KochConsciousness2019}.

% ------------------------------------------------------------------------------------------------

\section{Axiom 1: Semantic Manifold}
\label{1.1:axiom_1_semantic_manifold}

We propose that a differentiable manifold \(\mathcal{M}\) (semantic space), equipped with a dynamic metric tensor \(g_{\mu\nu}(p,t)\):

\begin{equation}
g_{\mu\nu}(p,t) : \mathcal{M} \times \mathbb{R} \rightarrow \mathbb{R}
\end{equation}

\begin{equation}
ds^2 = g_{\mu\nu}(p,t) \, dp^\mu \, dp^\nu
\end{equation}

Referred to as the \textit{Semantic Manifold}, this defines distances, curvature, and geodesics in meaning space, in concert with Riemannian geometry \autocite{Riemann1868}. Proximity and curvature, as well as the "pathways" between ideas, can be meaningfully quantified in this form. This builds on Peter Gärdenfors' work on conceptual spaces \autocite{Gardenfors2000}, in which he proposes \textit{meaning} as representable in geometric structure, and that acts of communication can be modeled as a topology \autocite{Gardenfors2014}. The manifold substrate evolves with the creation of meaningful connections, as the act of developing a new concept curves subsequent possibility space (its "semantic neighborhood") toward a more specific and coherent state.

With a manifoldic canvas established, the next step is to define the fields that populate it and encode the configuration of \textit{meaning}.

% ------------------------------------------------------------------------------------------------

\section{Axiom 2: Fundamental Semantic Field}
\label{1.2:axiom_2_fundamental_semantic_field}

We assert that a vector field \(\psi^\mu(p,t)\) on \(\mathcal{M}\) represents the semantic configuration. Coherence \(C^\mu(p,t)\) is a functional of \(\psi^\mu\).

\begin{equation}
C^\mu(p,t) = \mathcal{F}^\mu[\psi](p,t)
\end{equation}

\begin{equation}
C_{\text{mag}}(p,t) = \sqrt{g_{\mu\nu}(p,t) C^\mu(p,t) C^\nu(p,t)}
\end{equation}

The concept of a field of forces operating in a psychological or semantic space has precedent in the "lifespace" or psychological field suggested by Kurt Lewin in his book on social psychology \autocite{Lewin1951}. Here, we take a conceptual leap. Rather than treating \textit{meaning} as a discrete point in the manifold, we can regard it as a continuous, dynamic field capable of exhibiting both local and global structure. Much like a magnetic field in physics, this can vary in strength and direction across semantic space, allowing us to quantify its alignment and coherence at any point.

From this we can formalize meta-dynamics: how semantic fields influence one another through recursive feedback relationships of self-reference and interpretation. 

% ------------------------------------------------------------------------------------------------

\section{Axiom 3: Recursive Coupling}
\label{1.3:axiom_3_recursive_coupling}

We posit a rank-3 tensor \(R^\rho_{\mu\nu}(p,q,t)\) to quantify the influence of activity at point \(q\) on coherence at point \(p\) through self-referential processes:

\begin{equation}
R^\rho_{\mu\nu}(p,q,t) = \frac{\mathcal{D}^2 C^\rho(p,t)}{\mathcal{D} \psi^\mu(p) \mathcal{D} \psi^\nu(q)}
\end{equation}

The Recursive Coupling Tensor is a first class object in this theory. It formalizes the intuition that \textit{meaning} is constructed and reconstructed via self-reference. Coherence dynamics in any given location are shaped by reverberations of semantic shift elsewhere, as the manifold itself is holistically-coupled.

Recursion is always a complication, however in this light, it is \textit{the} mechanism that breathes complexity into semantic space. Here, Douglas Hofstadter's "strange loops" and "tangled hierarchies" of self-reference \autocite{Hofstadter1979} are made mathematically concrete. This sets a stage for everything from emergent agency to the potential for recursive pathologies, which we explore in Chapter \ref{16:pathologies_of_the_semantic_manifold}. Now, we are equipped to describe "mass" emergence.

% ------------------------------------------------------------------------------------------------

\section{Axiom 4: Geometric Coupling Principle}
\label{1.4:axiom_4_geometric_coupling_principle}

Semantic mass \(M(p,t)\) curves the manifold's geometry according to:

\begin{equation}
R_{\mu\nu} - \frac{1}{2}g_{\mu\nu}R = 8\pi G_s T^{\text{rec}}_{\mu\nu}
\end{equation}

We find that a Semantic Mass Equation is structurally analogous to the field equations of general relativity \autocite{Einstein1915, MisnerThorneWheeler1973, Wald1984}, where the recursive stress-energy tensor \(T^{\text{rec}}_{\mu\nu}\) is an analog of the mass-energy tensor in spacetime curvature. Here we define semantic mass as:

\begin{equation}
M(p,t) = D(p,t) \cdot \rho(p,t) \cdot A(p,t)
\end{equation}

\begin{equation}
\rho(p,t) = \frac{1}{\det(g_{\mu\nu}(p,t))}
\end{equation}

% ------------------------------------------------------------------------------------------------

\section{Axiom 5: Variational Evolution}
\label{1.5:axiom_5_variational_evolution}

We derive the dynamics of semantic fields from the principle of stationary action applied to the Lagrangian, where field dynamics preserve symmetries and conservation laws, consistent with the variational principle \autocite{GoldsteinPooleSafko2002, Arnold1989}.

\begin{equation}
\mathcal{L} = \frac{1}{2} g_{\mu\rho} g_{\nu\sigma} (\nabla^\rho C^\mu)(\nabla^\sigma C^\nu) - V(C_{\text{mag}}) + \Phi(C_{\text{mag}}) - \lambda_H \mathcal{H}[R]
\end{equation}

where

\begin{equation}
\frac{\delta S}{\delta C^\mu} = 0 \quad \text{and} \quad S = \int_{\mathcal{M}} \mathcal{L} \, dV
\end{equation}

% ------------------------------------------------------------------------------------------------

\section{Axiom 6: Autopoietic Threshold}
\label{1.6:axiom_6_autopoietic_threshold}

We recognize that when coherence magnitude exceeds a critical threshold, the autopoietic potential \(\Phi(C_{\text{mag}})\) becomes positive and drives generative phase transitions. This marks the point at which a system achieves a state of self-producing and self-maintaining autonomy, first defined by Humberto Maturana and Francisco J. Varela in their seminal work on theoretical biology \autocite{MaturanaVarela1980}.

The transition to this state is a physical phenomenon of self-organization common to complex systems. We derive the mathematical language for such phase transitions from the field of synergetics \autocite{Haken1983}, where macroscopic order emerges from the collective behavior of microscopic components. Furthermore, the emergence of such order is an expected property of sufficiently complex networks, which naturally exhibit self-organizing criticality \autocite{BakTangWiesenfeld1987}.

\begin{equation}
\Phi(C_{\text{mag}}) = \begin{cases}
\alpha_{\Phi} (C_{\text{mag}} - C_{\text{threshold}})^{\beta_{\Phi}} & \text{if } C_{\text{mag}} \geq C_{\text{threshold}} \\
0 & \text{otherwise}
\end{cases}
\end{equation}

% ------------------------------------------------------------------------------------------------

\section{Axiom 7: Recurgence}
\label{1.7:axiom_7_recurgence}

We define a semantic system to possess the capacity for \textit{recurgency} if it can autoreferentially model and reconfigure its own semantic structure through geometric evolution. \textit{Recurgence} is formally characterized by the non-zero second-order evolution of the metric tensor:

\begin{equation}
\frac{\partial^2 g_{\mu\nu}}{\partial t^2} \neq 0
\end{equation} 