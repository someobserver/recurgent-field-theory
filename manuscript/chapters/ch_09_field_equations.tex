\chapter{Field Equations}

\section{Overview}

Having established the components of the semantic manifold, we now present the field equations that govern its dynamics. These form a closed system describing the co-evolution of meaning and the geometric space it inhabits. The primary equations address three fundamental aspects of the system's evolution: (1) the propagation of coherence in response to recursive stress, (2) the natural "least-resistance" paths that semantic structures follow, and (3) the evolution of the manifold's metric itself. Together, they describe a cosmological model in which the flow of meaning is guided by the geometry, and the geometry is continually reshaped by the flow of meaning. The entire system is described using the formal language of partial differential equations, a standard and powerful tool for modeling complex continuous systems in physics and mathematics \autocite{Evans2010}.

\section{Recurgent Field Equation}

Coherence cannot evolve in isolation, but must dynamically respond to recursive stresses that pervade semantic space. Recursive structures generate stress-energy that shapes coherence evolution throughout the semantic manifold, analogous to how massive objects in general relativity curve spacetime and influence the motion of other masses.

The recurgent field equation captures this central dynamic:

\begin{equation}
\Box C_i = T^{\text{rec}}_{ij} \, g^{jk} C_k
\end{equation}

where \(\Box = \nabla^a \nabla_a\) denotes the covariant d'Alembertian operator on the semantic manifold \(\mathcal{M}\), \(T^{\text{rec}}_{ij}\) is the recursive stress-energy tensor, and \(g^{jk}\) is the inverse metric tensor.

This is structurally analogous to wave equations in field theory \autocite{Ryder1996, Weinberg1995}, in which the d'Alembertian operator governs field propagation. The equation purports that coherence acceleration (both spatially and temporally) is shaped by local recursive stress and semantic constraint geometry. In regions of elevated semantic mass or pronounced recursive torsion, the coherence field bends and may collapse into attractor basins. Where density is low, coherence spreads more diffusively, following the natural contours of semantic space.

\section{Conservation of the Recursive Stress-Energy Tensor}

For the recurgent field equation to be mathematically well-posed, the recursive stress-energy tensor must satisfy a fundamental conservation law. Such a requirement ensures that the field dynamics preserve essential physical quantities and maintain consistency with the underlying geometric structure of the semantic manifold.

The conservation law requires that the recursive stress-energy tensor be divergence-free:

\begin{equation}
\nabla_j T^{\mathrm{rec}}_{ij} = 0
\end{equation}

Recursive processes cannot create or destroy semantic "matter" arbitrarily, but must preserve total semantic energy-momentum within the manifold. Structurally, this follows energy-momentum conservation in general relativity \autocite{MisnerThorneWheeler1973, Wald1984}.

Establishing Conservation: Several approaches can establish this conservation law:

\begin{enumerate}
    \item Lagrangian Formalism: If \(T^{\mathrm{rec}}_{ij}\) derives from a matter Lagrangian \(\mathcal{L}_M\) for the semantic field,
    \begin{equation}
    T^{\mathrm{rec}}_{ij} = \frac{2}{\sqrt{-g}} \frac{\delta \mathcal{L}_M}{\delta g^{ij}}
    \end{equation}
    then Noether's theorem \autocite{Noether1918} regarding diffeomorphism invariance yields conservation automatically, provided the field equations hold.
    \item Modified Field Equations: If non-conserved components appear, the field equations can be supplemented with correction terms to preserve the Bianchi identities \autocite{Bianchi1902} and maintain internal consistency.
    \item Constraint Enforcement: In computational implementations, constraint forces (via Lagrange multipliers) may be introduced to maintain conservation numerically while preserving the essential dynamics.
\end{enumerate}

\section{Semantic Geodesics}

The natural trajectory of a semantic point \(p \in \mathcal{M}\) under recursive evolution is described by the geodesic equation:

\begin{equation}
\frac{d^2 p^i}{ds^2} + \Gamma^i_{jk} \frac{dp^j}{ds} \frac{dp^k}{ds} = 0
\end{equation}

where
\begin{itemize}
    \item \(s\) is a parameter along the curve (e.g., time or recursive depth),
    \item \(\Gamma^i_{jk}\) are the Christoffel symbols associated with the metric \(g_{ij}\),
    \item \(p^i(s)\) are the coordinates of the evolving semantic state.
\end{itemize}

Interpretation:

Geodesics trace the extremal (least-resistance) paths of recursive transformation on the manifold. The geodesic equation follows the standard form from differential geometry \autocite{doCarmo1992, Einstein1915, MisnerThorneWheeler1973} to describe extremal paths on curved manifolds. The curvature encoded by \(\Gamma^i_{jk}\) bends these paths, giving rise to semantic attractors. As such, the geodesic structure guides the spontaneous alignment of evolving meaning with established semantic trajectories.

\section{Metric Evolution}

The geometry of the semantic manifold is itself dynamic, evolving in response to recursive flows and the accumulation of semantic mass. The evolution of the metric tensor is given by:

\begin{equation}
\frac{\partial g_{ij}}{\partial t} = -2 R_{ij} + F_{ij}(R, D, A)
\end{equation}

where
\begin{itemize}
    \item \(R_{ij}\) is the Ricci curvature tensor, encoding the torsion induced by recursion,
    \item \(F_{ij}\) is a forcing term dependent on the recursive coupling \(R\), recursive depth \(D\), and attractor stability \(A\).
\end{itemize}

This follows the structure of Ricci flow \autocite{Hamilton1982, Perelman2002}, with additional forcing terms specific to recursive semantic dynamics.

Implication:

Recursive processes generate curvature in the semantic geometry, which then modulates subsequent recursive flows. This creates a closed feedback system in whichthe manifold's structure is continually reshaped by the propagation of coherence, even as it shapes that propagation in turn.

\section{Recursive Dynamical Structure}

The interdependence of the primary fields and their governing equations can be schematically represented as follows:

\begin{verbatim}
[ C(p,t) ] 
    ↓ (via ∇C)
[ T^{rec}_{ij} ] —→ drives —→ [ □C_i = T^{rec}_{ij} g^{jk} C_k ]
    ↓
[ R_{ij} ] ←— influenced by —→ [ dR/dt ∝ Φ(C) ]
    ↓
[ g_{ij} ] ←— updated by —→ [ ∂g_{ij}/∂t = -2R_{ij} + F ]
    ↓
[ ∇C, geodesics, box C ] ←— act back on —→ [ C(p,t) ]
\end{verbatim} 