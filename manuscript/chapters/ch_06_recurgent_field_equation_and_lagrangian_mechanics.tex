\chapter{Recurgent Field Equation and Lagrangian Mechanics}
\label{ch:recurgent_field_equation_and_lagrangian_mechanics}

% ------------------------------------------------------------------------------------------------

\section{Overview}

Semantic structures evolve according to the principle of stationary action, serving as the foundation for their dynamics. This principle, central to modern field theory \autocite{GoldsteinPooleSafko2002, Arnold1989}, forms a core tenet of our framework (\S\ref{sec:axiom_5}). The Lagrangian, a single scalar function, captures the interplay of competing semantic forces from which emerge the equations of motion. Here we present the specific Lagrangian for Recurgent Field Theory and derive the corresponding Euler-Lagrange field equation governing coherence evolution across the manifold.

% ------------------------------------------------------------------------------------------------

\section{Lagrangian Density}
\label{sec:lagrangian_density}

Semantic dynamics arise from a tension between coherence-seeking flow, the stabilizing influence of attractors, generative autopoietic potential, and regulatory constraints against pathological recursion. The Lagrangian density \(\mathcal{L}\) for a real coherence field \(C_i\) encodes these competing influences:
\begin{equation}
\mathcal{L} = \underbrace{\frac{1}{2} g^{ij} (\nabla_i C_k)(\nabla_j C^k)}_{\text{Kinetic Term}} - \underbrace{V(C_{\text{mag}})}_{\text{Potential}} + \underbrace{\Phi(C_{\text{mag}})}_{\text{Autopoiesis}} - \underbrace{\lambda \mathcal{H}[R]}_{\text{Constraint}}
\end{equation}
where summation over repeated indices is implied. In this manner, a macroscopic field (an order parameter, analogous to the coherence field \(C_i\)) is governed by a phenomenological Lagrangian whose potential landscape is engineered to produce a phase transition. Its origins lie in the theory of superconductivity, where it was used to describe the transition from a normal to a superconducting state \autocite{GinzburgLandau1950}. The components are:
\begin{itemize}
    \item \textbf{Kinetic Term:} The standard kinetic energy for a multicomponent field, which penalizes non-uniform coherence gradients.
    \item \textbf{Potential Term \(V(C_{\text{mag}})\):} A potential function that encodes the influence of stable semantic attractors, driving the system toward states of established meaning.
    \item \textbf{Autopoietic Term \(\Phi(C_{\text{mag}})\):} A generative potential that becomes active above a critical coherence threshold, driving the formation of novel semantic structures.
    \item \textbf{Humility Constraint \(\mathcal{H}[R]\):} A functional of the recursive coupling tensor \(R\) that provides a regulatory mechanism to penalize excessive or unstable recursive amplification. The parameter \(\lambda\) modulates its strength.
\end{itemize}
The potential, autopoietic, and humility terms, which encode these dynamics, are detailed in Chapters \ref{ch:attractor_dynamics}, \ref{ch:autopoiesis}, and \ref{ch:wisdom_humility}, respectively.

With this formulation, the resulting field equations are covariant. Any continuous symmetry in the Lagrangian gives rise to a corresponding conservation law, in accordance with Noether's theorem and the fundamental symmetries of theoretical physics \autocite{Noether1918, Lagrange1788, Euler1744, LandauLifshitz1975, PeskinSchroeder1995, Weinberg1995}.

% ------------------------------------------------------------------------------------------------

\subsection{Complex Field Formulation}
\label{sec:complex_field_formulation}
For systems with wave-like phenomena or phase dynamics, the coherence field must be complex-valued, requiring an extended Lagrangian:
\begin{equation}
\mathcal{L}_{\mathbb{C}} = g^{ij} (\nabla_i C_k)(\nabla_j C^{k*}) - V(|C|) + \Phi(|C|) - \lambda \mathcal{H}[R]
\end{equation}
where \(C^{k*}\) is the complex conjugate of \(C^k\) and \(|C| = \sqrt{g^{ij} C_i C_j^*}\). This formulation, analogous to that of Schrödinger or Dirac fields, models propagating semantic waves and interference effects.

% ------------------------------------------------------------------------------------------------

\section{The Principle of Stationary Action}
\label{sec:the_principle_of_stationary_action}

The action functional, \(S\), is the integral of the Lagrangian density over the Semantic Manifold \(\mathcal{M}\):

\begin{equation}
S[C_i] = \int_{\mathcal{M}} \mathcal{L}(C_i, \nabla_j C_i, R) \, dV
\end{equation}

where \(dV = \sqrt{|g|} \, d^n p\) is the invariant volume element. The principle of stationary action, \(\delta S = 0\), requires that the physical evolution of the field follow a path that extremizes this functional.

% ------------------------------------------------------------------------------------------------

\section{Euler–Lagrange Field Equation}
\label{sec:euler_lagrange_field_equation}

The variational principle, applied to the action \(S\), yields the Euler–Lagrange equations for the coherence field \(C_i\) \autocite{Euler1744, Lagrange1788}:

\begin{equation}
\frac{\partial \mathcal{L}}{\partial C_i} - \nabla_j \left( \frac{\partial \mathcal{L}}{\partial (\nabla_j C_i)} \right) = 0
\end{equation}

Substituting the components of \(\mathcal{L}\) yields the explicit equation of motion:

\begin{equation}
\Box C^i + \frac{\partial V(C_{\mathrm{mag}})}{\partial C_i} - \frac{\partial \Phi(C_{\mathrm{mag}})}{\partial C_i} + \lambda \frac{\partial \mathcal{H}[R]}{\partial C_i} = 0
\end{equation}
where \(\Box \equiv g^{jk}\nabla_j \nabla_k\) is the covariant d'Alembertian operator. The potential terms are functions of the coherence magnitude, \(C_{\text{mag}} = \sqrt{g^{ij} C_i C_j}\), and their derivatives are found via the chain rule:
\begin{equation}
\frac{\partial V(C_{\mathrm{mag}})}{\partial C_i} = \frac{dV}{dC_{\mathrm{mag}}} \frac{\partial C_{\mathrm{mag}}}{\partial C_i} = \frac{dV}{dC_{\mathrm{mag}}} \frac{g^{ij} C_j}{C_{\mathrm{mag}}}
\end{equation}
The humility term requires a functional derivative, since \(\mathcal{H}\) depends on the recursive coupling tensor \(R\), which is itself a functional of the underlying semantic field \(\psi\) that generates \(C\):
\begin{equation}
\frac{\partial \mathcal{H}[R]}{\partial C_i(p)} = \int_{\mathcal{M}} \frac{\delta \mathcal{H}[R]}{\delta R_{jkl}(s)} \frac{\delta R_{jkl}(s)}{\delta C_i(p)} \, dV_s
\end{equation}
This term represents a nonlocal feedback loop in which the global recursive structure influences local coherence dynamics.

% ------------------------------------------------------------------------------------------------

\section{Microscopic Dynamics and Field Coupling}
\label{sec:microscopic_dynamics_and_field_coupling}

The Euler-Lagrange equation for \(C_i\) provides the effective dynamics of coherence. However, the axiomatic foundation (Chapter 1) posits a more fundamental semantic field, \(\psi_i\), from which coherence emerges (\(C_i = \mathcal{F}_i[\psi]\)). A full description of the system requires that we specify the dynamics of \(\psi_i\) and its coupling to \(C_i\).

% ------------------------------------------------------------------------------------------------

\subsection{Semantic Field Evolution}
\label{sec:semantic_field_evolution}

We describe the evolution of the microscopic field \(\psi_i\) with a flow equation:
\begin{equation}
\frac{\partial \psi_i(p, t)}{\partial t} = v_i[\psi, C](p, t)
\end{equation}
The semantic velocity \(v_i\) is driven by gradients in the effective coherence landscape and other recursive forces. A general form for this velocity is:
\begin{equation}
v_i(p, t) = \alpha \cdot \nabla_i C_{\mathrm{mag}}(p, t) + \mathcal{G}_i[\psi](p, t)
\end{equation}
where:
\begin{itemize}
    \item The first term is gradient flow, in which \(\psi_i\) evolves to increase local coherence. \(\alpha\) is a coupling constant.
    \item The second term, \(\mathcal{G}_i[\psi]\), includes all other direct recursive forces and influences not mediated by the mean coherence field \(C\). Its specific form depends on the system being modeled.
\end{itemize}
This establishes a bidirectional, multi-scale coupling: microscopic variations in \(\psi_i\) determine the structure of the macroscopic coherence field \(C_i\), which in turn guides the evolution of \(\psi_i\).

% ------------------------------------------------------------------------------------------------

\subsection{The Coupled Dynamical System}
\label{sec:the_coupled_dynamical_system}

The complete theoretical structure comprises a coupled system of partial differential equations:
\begin{enumerate}
    \item \textbf{Microscopic Evolution:} \(\displaystyle \frac{\partial \psi_i}{\partial t} = v_i[\psi, C]\)
    \item \textbf{Macroscopic Definition:} \(C_i = \mathcal{F}_i[\psi]\)
    \item \textbf{Effective Field Equation:} \(\Box C^i + \frac{\partial V}{\partial C_i} - \frac{\partial \Phi}{\partial C_i} + \lambda \frac{\partial \mathcal{H}}{\partial C_i} = 0\)
\end{enumerate}
We may solve the system numerically by iterating between the levels: \(\psi_i\) is updated via its evolution equation, the resulting \(C_i\) is calculated, and \(C_i\) must satisfy the Euler-Lagrange equation. The underlying action principle guarantees the consistency of this procedure, provided the variation \(\delta C_i\) is constrained by admissible variations in \(\psi_i\):
\begin{equation}
\delta C_i(p) = \int_{\mathcal{M}} \frac{\delta C_i(p)}{\delta \psi_j(q)} \, \delta \psi_j(q) \, dV_q
\end{equation}
The dynamics derived from the effective Lagrangian for \(C_i\) therefore remain consistent with the evolution of the fundamental field \(\psi_i\).