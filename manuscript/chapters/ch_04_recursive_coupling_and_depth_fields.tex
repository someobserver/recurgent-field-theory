\chapter{Recursive Coupling and Depth Fields}
\label{ch:recursive_coupling_and_depth_fields}

% ------------------------------------------------------------------------------------------------

\section{Overview}

Self-reference is integral to the structure of meaning. The act of thinking about thinking, or using language to describe language, creates feedback loops that both stabilize and transform semantic structures. While often modeled as discrete graphs in network science \autocite{Barabasi2016}, we formalize these feedback mechanisms here with continuous tensor fields governing recursive processes. The interplay of these tensors generates forces that shape the manifold, leading to complexity and emergent patterns of thought. We define the core tensors quantifying their dynamics below.

% ------------------------------------------------------------------------------------------------

\section{The Recursive Coupling Tensor}
\label{sec:the_recursive_coupling_tensor}

The recursive coupling tensor, \(R_{ijk}(p, q, t)\), captures the non-local, bidirectional influence that semantic activity at one point exerts on another. It is the second-order variation of the coherence field with respect to the underlying semantic field, \(\psi\):
\begin{equation}
R_{ijk}(p, q, t) = \frac{\partial^2 C_k(p,t)}{\partial \psi_i(p) \partial \psi_j(q)}
\end{equation}
This tensor quantifies how a change in the semantic field component \(\psi_j\) at point \(q\) affects the sensitivity of the coherence component \(C_k\) at point \(p\) to changes in its own local semantic field, \(\psi_i\). Per Chapter 2, it possesses a dual character: both a measurement of the field's response properties and a dynamical field in its own right.

% ------------------------------------------------------------------------------------------------

\subsection{On Contrapuntal Coupling}
\label{sec:on_contrapuntal_coupling}

Counterpoint provides the mathematical principle underlying recursive coupling dynamics, finding cultural description in the works of Johann Sebastian Bach. A fugue begins with a simple melodic subject functioning as its concentrated semantic seed, with high recursive depth \(D(p,t)\). This subject propagates through the manifold as successive voices enter, each restating the theme at different points in semantic space. We can represent a voice entry as a coupling event where the subject appears at a new location \(q\) while maintaining bidirectional influence with all previous entries. Despite independent trajectories, contrapuntal voices remain bound to the whole in interdependence. Each conditions and is conditioned by every other voice through the landscape of the evolving harmonic field.

Bach's strictest mathematical rules enabled vast interpretive variance within bounded structure. This mirrors the constraint density we define on the Semantic Manifold \(\rho(p,t) = 1/\det(g_{ij})\), creating the conditions for innovation through geometric constraint. His works demonstrate the autopoietic potential \(\Phi(C_{\text{mag}})\) where sufficient coherence creates the conditions for self-generating semantic elaboration, most systematically explored in \textit{The Art of Fugue} \autocite{Bach1751}. Its development follows from the mathematical properties of the subject: once the initial semantic seed is established, recursive coupling dynamics generate the fine details of the structure through their inherent logic.

% ------------------------------------------------------------------------------------------------

\section{Recursive Depth}
\label{sec:recursive_depth}

The tensor \(R_{ijk}\) defines the mechanism of recursion; the depth field, \(D(p, t)\), quantifies its local sustainability. We define the scalar function \(D(p,t)\) as the maximal number of recursive layers a structure at point \(p\) can support before its coherence degrades below a functional threshold, \(\epsilon\):
\begin{equation}
D(p, t) = \max \left\{ d \in \mathbb{N} : \left\| \frac{\partial^d C(p,t)}{\partial \psi^d} \right\| \geq \epsilon \right\}
\end{equation}
where the norm is taken over the tensor indices of the higher-order derivative. Structures with high depth (e.g., persistent personal narratives) maintain coherence across many layers of self-reference, whereas those with low depth (e.g., simple arithmetic) have a shallow recursive structure.

This measure distinguishes meaningful, structured complexity from both trivial simplicity and incompressible randomness. A crystal is simple, a gas is random, but a living organism is deep. This is a direct implementation of "logical depth," which defines complexity not by the length of a description but by the computational time required to generate an object from its most compressed representation \autocite{Bennett1988}.

% ------------------------------------------------------------------------------------------------

\section{The Recursive Stress-Energy Tensor}
\label{sec:the_recursive_stress_energy_tensor}

The recursive stress-energy tensor, \(T_{ij}^{\text{rec}}\), quantifies the contribution of recursive activity to the curvature of the Semantic Manifold, analogous to the stress-energy tensor in general relativity \autocite{Einstein1915}. It captures the momentum and pressure of recursive processes.
\begin{equation}
T_{ij}^{\text{rec}} = \rho(p,t) v_i(p,t) v_j(p,t) + P_{ij}(p,t)
\end{equation}
where:
\begin{itemize}
    \item \(\rho(p,t)\) is the constraint density from the metric.
    \item \(v_i(p,t) = \frac{d\psi_i(p,t)}{dt}\) is the semantic velocity, the rate of change in the underlying semantic field.
    \item The recursive pressure tensor, \(P_{ij}(p,t)\), accounts for internal stresses within the semantic fluid caused by recursive flows. It takes the form:
\end{itemize}
\begin{equation}
P_{ij} = \gamma(\nabla_i v_j + \nabla_j v_i) - \eta g_{ij} (\nabla_k v^k)
\end{equation}
where \(\gamma\) is a shear viscosity (the elasticity of recursive loops) and \(\eta\) is a bulk viscosity (the resistance to isotropic recursive compression or expansion). The mathematical structure of this viscous pressure tensor is adopted directly from the classical theory of fluid mechanics \autocite{LandauLifshitz1987}.