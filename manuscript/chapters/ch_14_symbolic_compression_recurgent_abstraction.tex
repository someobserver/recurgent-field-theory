\chapter{Symbolic Compression and Abstraction}

\section{Overview}

A primary function of any advanced cognitive system is the ability to create abstractions by distilling vast and complex phenomena into compact, higher-order concepts (e.g., the concept of a "market" need not track every participant and transaction). This process represents a thermodynamic and computational necessity for managing the complexity of recursive systems. This section formalizes abstraction in RFT through semantic compression operators. These operators reduce a semantic structure's dimensionality while preserving its essential dynamical and structural properties, creating the hierarchical manifolds of meaning characteristic of sophisticated thought. The resulting formalism aligns with algorithmic information theory's principle that object complexity is measured by the length of its shortest possible description \autocite{Kolmogorov1965, Chaitin1966}. An information-centric perspective on cognitive structure also provides a bridge to theories grounding consciousness in the mathematics of information integration \autocite{Tononi2004}. Ultimately, this resonates with hypotheses of the physical world itself being fundamentally informational, famously articulated as "it from bit" \autocite{Wheeler1990}.

\section{Semantic Compression Operators}

Abstraction is an operator, \(\mathcal{C}\), taking a submanifold of meaning, \(\Omega \subset \mathcal{M}\), and producing a new, lower-dimensional submanifold, \(\Omega' \subset \mathcal{M}'\), with \(\dim(\mathcal{M}') < \dim(\mathcal{M})\).
\begin{equation}
\mathcal{C}: \Omega \subset \mathcal{M} \longrightarrow \Omega' \subset \mathcal{M}'
\end{equation}
A valid and useful abstraction must preserve the core essence of the original structure. In RFT, a valid compression operator, \(\mathcal{C}\), must satisfy four structural invariants. These conditions are direct consequences of the theory's foundational principles of conservation and stability.

\subsection{The Four Invariants of Semantic Compression}

\begin{enumerate}
    \item \textbf{Coherence Preservation:} The total coherence of a concept must be approximately conserved; an abstraction must capture the same "amount" of meaning as the original.
    \begin{equation}
    \int_{\Omega} C_{\text{mag}}(p) \, dV_p \approx \int_{\Omega'} C'_{\text{mag}}(p') \, dV'_{p'}
    \end{equation}
    The compressed concept thereby remains as meaningful as the original.

    \item \textbf{Recursive Integrity:} The net recursive flux across the boundary of the conceptual domain must be preserved. Analogous to Gauss's Law, this ensures the abstracted concept has the same net generative or consumptive relationship with its environment.
    \begin{equation}
    \oint_{\partial \Omega} F_i \, dS^i \approx \oint_{\partial \Omega'} F'_i \, dS'^i
    \end{equation}
    where \(F_i = -\nabla_i V(p,t)\) is the recursive force field from Chapter 5.

    \item \textbf{Wisdom Concentration:} The mean wisdom density must be non-decreasing. A valid abstraction must be at least as wise as the structure from which it was derived.
    \begin{equation}
    \frac{\int_{\Omega} W(p) \, dV_p}{\operatorname{Vol}(\Omega)} \leq \frac{\int_{\Omega'} W'(p') \, dV'_{p'}}{\operatorname{Vol}(\Omega')}
    \end{equation}
    Governed by the wisdom field \(W(p,t)\) (Chapter 8), this constraint prevents the formation of "foolish" or brittle abstractions that otherwise discard critical regulatory intuition.

    \item \textbf{Metric Congruence:} The geometry of the abstracted space must be consistent with the original. Formally, a diffeomorphism \(\phi: \Omega' \to \Omega\) must exist such that the compressed metric \(g'_{ij}\) is approximately the pullback of the original metric \(g_{ij}\).
    \begin{equation}
    g'_{ij}(p') \approx (\phi^*g)_{ij} = \frac{\partial \phi^k}{\partial x'^i} \frac{\partial \phi^l}{\partial x'^j} g_{kl}(\phi(p'))
    \end{equation}
    The relationships and distances between concepts are thereby preserved in the abstraction.
\end{enumerate}

\section{Hierarchical Manifolds}

The repeated application of semantic compression operators generates a hierarchy of nested semantic manifolds:
\begin{equation}
\mathcal{M}_0 \supset \mathcal{M}_1 \supset \cdots \supset \mathcal{M}_N
\end{equation}
Each manifold \(\mathcal{M}_k\) represents a distinct level of abstraction, with lower dimensionality and greater semantic generality than the level below it (\(\mathcal{M}_{k-1}\)). The hierarchy permits a cognitive system to move fluidly between concrete, high-dimensional representations and abstract, low-dimensional ones without losing theoretical consistency. A compression operator \(\mathcal{C}_k\) satisfying the four invariants achieves the transition from one level to the next, \(\mathcal{M}_k \to \mathcal{M}_{k+1}\). This multi-resolution geometry provides a formal basis for reasoning at multiple levels of abstraction simultaneously.
