\chapter{Semantic Mass and Attractor Dynamics}

\section{Overview}

The semantic mass equation is the cornerstone of Recurgent Field Theory. It quantifies the capacity of a meaning structure to influence its local environment and shape the geometry of the manifold. That would be analogous to mass-energy in general relativity: just like mass curves spacetime, semantic mass curves possibility-space toward stable attractors. The curvature is governed by a field equation linking the geometry to the recursive stress-energy of the field. Regions of high semantic mass function as stable attractors, creating basins that guide Ricci flow and anchor interpretation. The result is a dynamic landscape; the accumulation of meaning generates the very structure it inhabits.

\section{Semantic Mass}

Mass in RFT quantifies the capacity of meaning structures to shape manifold geometry. Semantic mass combines three fundamental factors multiplicatively because weakness in any component undermines the overall mass effect:
\begin{equation}
M(p, t) = D(p, t) \cdot \rho(p, t) \cdot A(p, t)
\end{equation}

where \(D(p, t)\) quantifies the maximal recursion depth sustainable at \(p\) before coherence degrades, \(\rho(p, t) = 1/\det(g_{ij}(p, t))\) encodes the tightness of local semantic geometry, and \(A(p, t)\) measures the local tendency of a semantic state to return after perturbation.

Semantic mass determines how powerfully a semantic structure influences its surroundings. Regions of high \(M\) are stable attractors, exerting a stabilizing influence on coherence field evolution and modulating recursive process propagation. The persistence of high-mass structures follows from their recursive depth, constraint density, and local stability, independent of their propositional content.

\section{Recurgent Einstein Equation}

The coupling between recursive stress and semantic curvature is governed by the recurgent Einstein field equation, directly parallelling his original from General Relativity \autocite{Einstein1915, MisnerThorneWheeler1973}:
\begin{equation}
R_{ij} - \frac{1}{2}g_{ij}R = 8\pi G_s T^{\text{rec}}_{ij}
\end{equation}

where:
\begin{itemize}
    \item \(R_{ij}\) is the Ricci curvature tensor of the semantic manifold,
    \item \(R\) is the scalar curvature,
    \item \(g_{ij}\) is the metric tensor,
    \item \(T^{\text{rec}}_{ij}\) is the recursive stress-energy tensor (see Section 4.3),
    \item \(G_s\) is the semantic gravitational constant.
\end{itemize}

This equation expresses the principle that recursive tension and constraint, as encoded in \(T^{\text{rec}}_{ij}\), generate curvature in semantic space, shaping the geometry of meaning in direct analogy to the role of mass-energy in general relativity.

\section[Attractor Potential Field Phi(p, t)]{Attractor Potential Field \(\Phi(p, t)\)}

The attractor potential field \(\Phi(p, t)\) is defined as the integral over semantic mass, weighted by geodesic distance:
\begin{equation}
\Phi(p, t) = -G_s \int_{\mathcal{M}} \frac{M(q, t)}{d(p, q)} \, dV_q
\end{equation}

where:
\begin{itemize}
    \item \(d(p, q)\) is the geodesic distance between points \(p\) and \(q\) in the manifold,
    \item \(M(q, t)\) is the semantic mass at \(q\),
    \item \(dV_q\) is the volume element.
\end{itemize}

The gradient of this potential gives the recursive force:
\begin{equation}
F_i(p, t) = -\nabla_i \Phi(p, t)
\end{equation}

which directs the flow of coherence and draws new semantic structures into existing attractor basins. Regions of high semantic mass modulate the dynamics of meaning, pulling recursive processes toward stable configurations.

\section{Potential Energy of Coherence}

The potential energy associated with the scalar coherence magnitude \(C_{\text{mag}}\) is given by:
\begin{equation}
V(C_{\text{mag}}) = \frac{1}{2}k(C_{\text{mag}} - C_0)^2
\end{equation}

where:
\begin{itemize}
    \item \(C_{\text{mag}} = \sqrt{g^{ij}(p, t) C_i(p, t) C_j(p, t)}\) is the scalar magnitude of the coherence field,
    \item \(C_0\) is the equilibrium coherence level of the attractor,
    \item \(k\) is the coherence rigidity parameter, quantifying the stiffness of the attractor basin.
\end{itemize}

This quadratic potential models the energetic landscape of attractors:
\begin{itemize}
    \item Soft attractors (e.g., metaphoric or fluid conceptual structures) correspond to small \(k\),
    \item Hard attractors (e.g., axiomatic, rigid, or dogmatic structures) correspond to large \(k\).
\end{itemize}

The parameter \(k\) modulates the resistance of an attractor to perturbation and the rate at which coherence returns to equilibrium following displacement. 