\chapter{Field Index and Formal Architecture}
\label{2:field_index_and_formal_architecture}

% ------------------------------------------------------------------------------------------------

\section{Overview}
\label{2.1:overview}

We express the theory in tensor calculus, and as such, each mathematical object corresponds to a geometric component of semantic reality. Its fields, tensors, and notations are drawn from differential geometry \autocite{Riemann1868, Lee2003}.

% ------------------------------------------------------------------------------------------------

\section{Tensor Ranks and Properties}
\label{2.2:tensor_ranks_and_properties}

The framework we set forth is pseudo-Riemannian and constructed on an \(n\)-dimensional manifold \(\mathcal{M}\), referred to as the \textit{Semantic Manifold}. The tensor rank and symmetry properties of each field encode its geometric information, while its domain and range encode its semantic content. The metric tensor \(g_{\mu\nu}\) establishes the foundational structure (\S\ref{1.1:axiom_1_semantic_manifold}). The semantic and coherence fields, \(\psi^\mu\) and \(C^\mu\), provide the dynamic content (\S\ref{1.2:axiom_2_fundamental_semantic_field}), while third- and fourth-rank tensors mediate the feedback loops that drive manifoldevolution (\S\ref{1.3:axiom_3_recursive_coupling}). All tensor expressions employ the Einstein summation convention, detailed in Section \ref{2.4:tensor_conventions_and_notation}.

The mathematical objects are organized into six thematic categories, reflecting the conceptual hierarchy of the theory.

% ------------------------------------------------------------------------------------------------

\subsection{Fundamental Fields and Geometric Structure}
\label{2.2.1:fundamental_fields_and_geometric_structure}

{\small
\renewcommand{\arraystretch}{1.1}
\begin{longtable}{|c|p{5.5cm}|c|c|p{1.8cm}|c|c|}
\hline
\textbf{Symbol} & \textbf{Name} & \textbf{Rank} & \textbf{Symmetry} & \textbf{Domain} & \textbf{Range} & \textbf{Dim} \\
\hline
\endfirsthead
\hline
\textbf{Symbol} & \textbf{Name} & \textbf{Rank} & \textbf{Symmetry} & \textbf{Domain} & \textbf{Range} & \textbf{Dim} \\
\hline
\endhead
\(g_{\mu\nu}(p,t)\) & Metric tensor & (0,2) & Sym & \(\mathcal{M} \times \mathbb{R}\) & \(\mathbb{R}\) & \(n^2\) \\
\hline
\(\psi^\mu(p,t)\) & Semantic field & (1,0) & - & \(\mathcal{M} \times \mathbb{R}\) & \(T_p\mathcal{M}\) & \(n\) \\
\hline
\(C^\mu(p,t)\) & Coherence vector field & (1,0) & - & \(\mathcal{M} \times \mathbb{R}\) & \(T_p\mathcal{M}\) & \(n\) \\
\hline
\(\Gamma^\rho_{\mu\nu}\) & Christoffel symbols & (1,2) & Sym(\(\mu\),\(\nu\)) & \(\mathcal{M}\) & \(\mathbb{R}\) & \(n^3\) \\
\hline
\(v^\mu\) & Semantic velocity & (1,0) & - & \(\mathcal{M} \times \mathbb{R}\) & \(T_p\mathcal{M}\) & \(n\) \\
\hline
\caption{Fundamental Fields and Geometric Structure}
\end{longtable}
}

% ------------------------------------------------------------------------------------------------

\subsection{Curvature and Geometric Quantities}
\label{2.2.2:curvature_and_geometric_quantities}

{\small
\renewcommand{\arraystretch}{1.1}
\begin{longtable}{|c|p{5.5cm}|c|c|p{1.8cm}|c|c|}
\hline
\textbf{Symbol} & \textbf{Name} & \textbf{Rank} & \textbf{Symmetry} & \textbf{Domain} & \textbf{Range} & \textbf{Dim} \\
\hline
\endfirsthead
\hline
\textbf{Symbol} & \textbf{Name} & \textbf{Rank} & \textbf{Symmetry} & \textbf{Domain} & \textbf{Range} & \textbf{Dim} \\
\hline
\endhead
\(R^{\rho}_{\sigma\mu\nu}\) & Riemann curvature tensor & (1,3) & Anti(\(\mu\),\(\nu\)) & \(\mathcal{M}\) & \(\mathbb{R}\) & \(n^4\) \\
\hline
\(R_{\mu\nu}\) & Ricci curvature tensor & (0,2) & Sym & \(\mathcal{M} \times \mathbb{R}\) & \(\mathbb{R}\) & \(n^2\) \\
\hline
\(G_{\rho\mu\nu}\) & Geometric structure tensor & (0,3) & Sym(\(\mu\),\(\nu\)) & - & \(\mathbb{R}\) & \(n^3\) \\
\hline
\(\kappa_t\) & Temporal curvature & 0 & - & \(\mathbb{R}^+\) & 1 & 1 \\
\hline
\caption{Curvature and Geometric Quantities}
\end{longtable}
}

% ------------------------------------------------------------------------------------------------

\subsection{Recursive Coupling and Feedback Dynamics}
\label{2.2.3:recursive_coupling_and_feedback_dynamics}

{\small
\renewcommand{\arraystretch}{1.1}
\begin{longtable}{|c|p{5.5cm}|c|c|p{1.8cm}|c|c|}
\hline
\textbf{Symbol} & \textbf{Name} & \textbf{Rank} & \textbf{Symmetry} & \textbf{Domain} & \textbf{Range} & \textbf{Dim} \\
\hline
\endfirsthead
\hline
\textbf{Symbol} & \textbf{Name} & \textbf{Rank} & \textbf{Symmetry} & \textbf{Domain} & \textbf{Range} & \textbf{Dim} \\
\hline
\endhead
\(R^\rho_{\mu\nu}(p,q,t)\) & Recursive coupling tensor & (1,2) & - & \(\mathcal{M}^2 \times \mathbb{R}\) & \(\mathbb{R}\) & \(n^3\) \\
\hline
\(\chi^\rho_{\mu\nu}(p,q,t)\) & Latent recursive channel tensor & (1,2) & - & \(\mathcal{M}^2 \times \mathbb{R}\) & \(\mathbb{R}\) & \(n^3\) \\
\hline
\(R^{\rho, \text{hetero}}_{\mu\nu}\) & Hetero-recursive tensor & (1,2) & - & \(\mathcal{M}^2 \times \mathbb{R}\) & \(\mathbb{R}\) & \(n^3\) \\
\hline
\(R^{(n)}\) & Meta-recursive tensor & (n,2n) & - & \(\mathcal{M}^n \times \mathbb{R}\) & \(\mathbb{R}\) & \(n^{3n}\) \\
\hline
\(S_{\mu\nu}(p,q)\) & Semantic similarity tensor\footnotemark[2] & (0,2) & Sym & \(\mathcal{M}^2\) & \(\mathbb{R}\) & \(n^2\) \\
\hline
\(H(p,q,t)\) & Historical co-activation\footnotemark[3] & 0 & - & \(\mathcal{M}^2 \times \mathbb{R}\) & \(\mathbb{R}^+\) & 1 \\
\hline
\(D^\rho_{\mu\nu}(p,q)\) & Domain incompatibility tensor & (1,2) & - & \(\mathcal{M}^2\) & \(\mathbb{R}^+\) & \(n^3\) \\
\hline
\(D(p,t)\) & Recursive depth & 0 & - & \(\mathcal{M} \times \mathbb{R}\) & \(\mathbb{N}\) & 1 \\
\hline
\caption{Recursive Coupling and Feedback Dynamics}
\end{longtable}
}

% ------------------------------------------------------------------------------------------------

\subsection{Physical Quantities and Dynamical Forces}
\label{2.2.4:physical_quantities_and_dynamical_forces}

{\small
\renewcommand{\arraystretch}{1.1}
\begin{longtable}{|c|p{5.5cm}|c|c|p{1.8cm}|c|c|}
\hline
\textbf{Symbol} & \textbf{Name} & \textbf{Rank} & \textbf{Symmetry} & \textbf{Domain} & \textbf{Range} & \textbf{Dim} \\
\hline
\endfirsthead
\hline
\textbf{Symbol} & \textbf{Name} & \textbf{Rank} & \textbf{Symmetry} & \textbf{Domain} & \textbf{Range} & \textbf{Dim} \\
\hline
\endhead
\(M(p,t)\) & Semantic mass & 0 & - & \(\mathcal{M} \times \mathbb{R}\) & \(\mathbb{R}^+\) & 1 \\
\hline
\(F_\mu(p,t)\) & Recursive force & (0,1) & - & \(\mathcal{M} \times \mathbb{R}\) & \(T_p^*\mathcal{M}\) & \(n\) \\
\hline
\(F_\mu^{\text{diss}}\) & Dissipative force & (0,1) & - & \(\partial\mathcal{A}\) & \(T_p^*\mathcal{M}\) & \(n\) \\
\hline
\(T_{\mu\nu}^{\text{rec}}\) & Recursive stress-energy tensor & (0,2) & Sym & \(\mathcal{M} \times \mathbb{R}\) & \(\mathbb{R}\) & \(n^2\) \\
\hline
\(P_{\mu\nu}\) & Recursive pressure tensor & (0,2) & Sym & \(\mathcal{M} \times \mathbb{R}\) & \(\mathbb{R}\) & \(n^2\) \\
\hline
\(F_{\mu\nu}\) & Metric forcing term & (0,2) & Sym & \(\mathcal{M}\) & \(\mathbb{R}\) & \(n^2\) \\
\hline
\(\rho(p,t)\) & Constraint density & 0 & - & \(\mathcal{M} \times \mathbb{R}\) & \(\mathbb{R}^+\) & 1 \\
\hline
\(\rho_V\) & Validation density & 0 & - & \(\mathcal{M} \times \mathbb{R}\) & \(\mathbb{R}^+\) & 1 \\
\hline
\caption{Physical Quantities and Dynamical Forces}
\end{longtable}
}

% ------------------------------------------------------------------------------------------------

\subsection{Potentials, Stability, and Phase Dynamics}
\label{2.2.5:potentials_stability_and_phase_dynamics}

{\small
\renewcommand{\arraystretch}{1.1}
\begin{longtable}{|c|p{5.5cm}|c|c|p{1.8cm}|c|c|}
\hline
\textbf{Symbol} & \textbf{Name} & \textbf{Rank} & \textbf{Symmetry} & \textbf{Domain} & \textbf{Range} & \textbf{Dim} \\
\hline
\endfirsthead
\hline
\textbf{Symbol} & \textbf{Name} & \textbf{Rank} & \textbf{Symmetry} & \textbf{Domain} & \textbf{Range} & \textbf{Dim} \\
\hline
\endhead
\(\Phi(C)\) & Autopoietic potential & 0 & - & \(T\mathcal{M}\) & \(\mathbb{R}^+\) & 1 \\
\hline
\(V(C)\) & Attractor potential\footnotemark[1] & 0 & - & \(T\mathcal{M}\) & \(\mathbb{R}^+\) & 1 \\
\hline
\(A(p,t)\) & Attractor stability\footnotemark[1] & 0 & - & \(\mathcal{M} \times \mathbb{R}\) & \([0,1]\) & 1 \\
\hline
\(\Theta(p,t)\) & Phase order parameter & 0 & - & \(\mathcal{M} \times \mathbb{R}\) & \(\mathbb{R}\) & 1 \\
\hline
\(W(p,t)\) & Wisdom field & 0 & - & \(\mathcal{M} \times \mathbb{R}\) & \(\mathbb{R}^+\) & 1 \\
\hline
\(\mathcal{H}[R]\) & Humility operator & 0 & - & \(\mathbb{R}\) & \(\mathbb{R}^+\) & 1 \\
\hline
\(S_R\) & Recurgence stability parameter & 0 & - & \(\mathcal{M} \times \mathbb{R}\) & \(\mathbb{R}^+\) & 1 \\
\hline
\(\mathcal{E}(t)\) & Recurgent expansion rate & 0 & - & \(\mathbb{R}\) & \(\mathbb{R}\) & 1 \\
\hline
\(S_{\text{sem}}\) & Semantic entropy & 0 & Func & \(P(\mathcal{M})\) & \(\mathbb{R}^+\) & 1 \\
\hline
\(\Gamma(\Omega)\) & Wisdom-coherence ratio & 0 & Func & \(P(\mathcal{M})\) & \(\mathbb{R}^+\) & 1 \\
\hline
\caption{Potentials, Stability, and Phase Dynamics}
\end{longtable}
}

% ------------------------------------------------------------------------------------------------

\subsection{Agent Fields and Communication Structures}
\label{2.2.6:agent_fields_and_communication_structures}

{\small
\renewcommand{\arraystretch}{1.1}
\begin{longtable}{|c|p{5.5cm}|c|c|p{1.8cm}|c|c|}
\hline
\textbf{Symbol} & \textbf{Name} & \textbf{Rank} & \textbf{Symmetry} & \textbf{Domain} & \textbf{Range} & \textbf{Dim} \\
\hline
\endfirsthead
\hline
\textbf{Symbol} & \textbf{Name} & \textbf{Rank} & \textbf{Symmetry} & \textbf{Domain} & \textbf{Range} & \textbf{Dim} \\
\hline
\endhead
\(\vec{P}^\mu\) & Proposition field & (1,0) & - & \(\mathcal{M} \times \mathbb{R}\) & \(T_p\mathcal{M}\) & \(n\) \\
\hline
\(\vec{V}_\mu\) & Validation field & (0,1) & - & \(\mathcal{M} \times \mathbb{R}\) & \(T_p^*\mathcal{M}\) & \(n\) \\
\hline
\(I^\mu\) & Interpretive field & (1,0) & - & \(\mathcal{M} \times \mathbb{R}\) & \(T_p\mathcal{M}\) & \(n\) \\
\hline
\(S_A\) & Agent attention field & 0 & - & \(\mathcal{M} \times \mathbb{R}\) & \([0,1]\) & 1 \\
\hline
\(N^\mu\) & Basis projection vector & (1,0) & - & - & \(T_p\mathcal{M}\) & \(n\) \\
\hline
\(T_{\mu\nu}^{(d \to d')}\) & Domain translation tensor & (0,2) & - & \(T\mathcal{M}_d \to T\mathcal{M}_{d'}\) & \(\mathbb{R}\) & \(n^2\) \\
\hline
\caption{Agent Fields and Communication Structures}
\end{longtable}
}

% ------------------------------------------------------------------------------------------------

\subsection{Operators, Functionals, and Constants}
\label{2.2.7:operators_functionals_and_constants}

{\small
\renewcommand{\arraystretch}{1.1}
\begin{longtable}{|c|p{5.5cm}|c|c|p{1.8cm}|c|c|}
\hline
\textbf{Symbol} & \textbf{Name} & \textbf{Rank} & \textbf{Symmetry} & \textbf{Domain} & \textbf{Range} & \textbf{Dim} \\
\hline
\endfirsthead
\hline
\textbf{Symbol} & \textbf{Name} & \textbf{Rank} & \textbf{Symmetry} & \textbf{Domain} & \textbf{Range} & \textbf{Dim} \\
\hline
\endhead
\(\Box\) & Covariant d'Alembertian & Op & \(C^2(\mathcal{M})\) & \(C^0(\mathcal{M})\) & - & - \\
\hline
\(\Delta_g\) & Laplace-Beltrami operator & Op & \(C^2(\mathcal{M})\) & \(C^0(\mathcal{M})\) & - & - \\
\hline
\(\nabla_f\) & Semantic forecast operator & Op & \(T\mathcal{M}\) & \(T\mathcal{M}\) & - & - \\
\hline
\(\mathcal{I}_{\psi}\) & Interpretive operator & Op & \(C^1(\mathcal{M})\) & \(C^1(\mathcal{M})\) & - & - \\
\hline
\(\mathcal{C}\) & Semantic compression operator & Op & \(P(\mathcal{M})\) & \(P(\mathcal{M}')\) & - & - \\
\hline
\(\mathcal{G}_\mu[\psi]\) & Recursive force functional & Func & \(C^1(\mathcal{M})\) & \(T_p^*\mathcal{M}\) & - & - \\
\hline
\hline
\multicolumn{7}{|c|}{\textbf{Physical Constants and Parameters}} \\
\hline
\(G_s\) & Semantic gravitational constant & 0 & - & \(\mathbb{R}^+\) & 1 & - \\
\hline
\(\gamma, \eta\) & Viscosity parameters & 0 & - & \(\mathbb{R}^+\) & 1 & - \\
\hline
\(k_V\) & Coherence rigidity & 0 & - & \(\mathbb{R}^+\) & 1 & - \\
\hline
\(\lambda_H\) & Humility strength & 0 & - & \(\mathbb{R}^+\) & 1 & - \\
\hline
\(\alpha_{\psi}\) & Microscopic coupling constant & 0 & - & \(\mathbb{R}\) & 1 & - \\
\hline
\(\alpha_{\Phi}\) & Autopoietic coupling constant & 0 & - & \(\mathbb{R}^+\) & 1 & - \\
\hline
\(\beta_{\Phi}\) & Critical exponent & 0 & - & \(\mathbb{R}^+\) & 1 & - \\
\hline
\(C_{\text{thr}}\) & Coherence threshold & 0 & - & \(\mathbb{R}^+\) & 1 & - \\
\hline
\(\hbar_s\) & Semantic uncertainty constant & 0 & - & \(\mathbb{R}^+\) & 1 & - \\
\hline
\caption{Operators, Functionals, and Constants}
\end{longtable}
}

\footnotetext[1]{The use of attractor stability metrics and potential energy landscapes for system characterization is drawn from nonlinear dynamics \autocite{Strogatz2014}.}

\footnotetext[2]{A formalization of the \textit{distributional hypothesis} in linguistics, which posits that words with similar distributions have similar meanings \autocite{Harris1954}. Other modern vector-space models of semantics, such as the Word2Vec framework, are built on this principle \autocite{Mikolov2013}.}

\footnotetext[3]{This serves as an implementation of Hebbian learning, which states that repeated, persistent co-activation of connected elements leads to an increase in the strength of their connection \autocite{Hebb1949}.}

% ------------------------------------------------------------------------------------------------

\section{System Architecture}
\label{2.3:system_architecture}

Coherence dynamics emerge from the interplay of four conceptual subsystems:

\begin{itemize}

    \item A geometric engine governs the evolution of the manifold's metric and curvature.

    \item A coherence processor handles the evolution of the primary fields.

    \item A recursive controller manages the coupling dynamics that link different regions of the manifold.

    \item A regulatory system provides wisdom and humility constraints.

\end{itemize}

The subsystems are deeply integrated and form two primary, coupled cycles. In the main causal loop, the coherence field determines a recursive stress-energy tensor, which in turn induces curvature in the metric. The deformed metric then governs the subsequent evolution of coherence, closing the primary feedback loop. 

Once coherence surpasses a critical threshold, a secondary generative cycle activates. This cycle uses the autopoietic potential to form new recursive pathways, thereby driving genuine structural innovation. The entire system is modulated by the regulatory subsystem, which employs the wisdom field and humility operator to prevent pathological amplification and maintain dynamic equilibrium.

% ------------------------------------------------------------------------------------------------

\section{Tensor Conventions and Notation}
\label{2.4:tensor_conventions_and_notation}

The tensor conventions follow modern standards for differential geometry and tensor calculus on smooth manifolds \autocite{Lee2003, MisnerThorneWheeler1973}. The tensor calculus framework from which this originates is the pioneering work of Gregorio Ricci Curbastro and Tullio Levi-Civita \autocite{RicciLeviCivita1901}.

% ------------------------------------------------------------------------------------------------

\subsection{Index Notation and Einstein Summation}
\label{2.4.1:index_notation_and_einstein_summation}

The Einstein summation convention \autocite{Einstein1916} applies, where repeated indices (one upper, one lower) imply summation:

\begin{equation}
A_\mu B^\mu = \sum_{\mu=1}^n A_\mu B^\mu
\end{equation}

Greek indices \((\mu,\nu,\rho,...)\) range from \(1\) to \(n\), the dimension of the Semantic Manifold.

% ------------------------------------------------------------------------------------------------

\subsection{Metric and Index Raising/Lowering}
\label{2.4.2:metric_and_index_raising_lowering}

The metric tensor \(g_{\mu\nu}\) and its inverse \(g^{\mu\nu}\) raise and lower indices (\(C^\mu = g^{\mu\nu} C_\nu\), \(C_\mu = g_{\mu\nu} C^\nu\)), satisfying \(g_{\mu\rho} g^{\rho\nu} = \delta_\mu^\nu\).

% ------------------------------------------------------------------------------------------------

\subsection{Covariant Derivatives}
\label{2.4.3:covariant_derivatives}

The covariant derivative \(\nabla_\mu\), defined via the Christoffel symbols \(\Gamma^\rho_{\mu\nu}\) \autocite{Christoffel1869}, accommodates the curved geometry of \(\mathcal{M}\):

\begin{equation}
\nabla_\mu C_\nu = \partial_\mu C_\nu - \Gamma^\rho_{\mu\nu} C_\rho \quad \text{and} \quad \Gamma^\rho_{\mu\nu} = \frac{1}{2} g^{\rho\sigma} ( \partial_\mu g_{\nu\sigma} + \partial_\nu g_{\mu\sigma} - \partial_\sigma g_{\mu\nu} )
\end{equation}

% ------------------------------------------------------------------------------------------------

\subsection{Functional and Variational Derivatives}
\label{2.4.4:functional_and_variational_derivatives}

The dynamics derive from an action principle, \(S = \int \mathcal{L} \, dV\), requiring variational derivatives. The Euler-Lagrange equations take the form:

\begin{equation}
\frac{\delta \mathcal{L}}{\delta C_\mu} = \frac{\partial \mathcal{L}}{\partial C_\mu} - \sum_\nu \nabla_\nu \left( \frac{\partial \mathcal{L}}{\partial (\nabla_\nu C_\mu)} \right)
\end{equation}

% ------------------------------------------------------------------------------------------------

\subsection{Integration and Symmetries}
\label{2.4.5:integration_and_symmetries}

Integrals over the manifold use the invariant volume element, \(dV = \sqrt{|\det(g_{\mu\nu})|} \, d^n p\). Tensor symmetries (e.g., \(g_{\mu\nu} = g_{\nu\mu}\)) are assumed and exploited where appropriate.

% ------------------------------------------------------------------------------------------------

\subsection{Fundamental versus Derived Fields}
\label{2.4.6:fundamental_versus_derived_fields}

We distinguish between the fundamental state of the system and its measured coherence:

\begin{itemize}

    \item The \textbf{semantic field} \(\psi^\mu(p,t)\) represents the raw, underlying semantic content at each point. It is the fundamental dynamical variable.

    \item The \textbf{coherence field} \(C^\mu(p,t)\) is a derived quantity measuring the self-consistency and alignment of the underlying semantic field. As an \(n\)-dimensional vector field, each of its components represents coherence along a principal semantic axis. It is a functional of \(\psi^\mu\):

\end{itemize}

\begin{equation}
C^\mu(p,t) = \mathcal{F}^\mu[\psi](p,t) = \int_{\mathcal{N}(p)} K^\mu_{\ \nu}(p,q) \psi^\nu(q,t) \, dq
\end{equation}

where \(K^\mu_{\ \nu}(p,q)\) is a non-local kernel. While the dynamics could be expressed in terms of \(\psi^\mu\), the Lagrangian is formulated using \(C^\mu\) to maintain a direct connection to semantic coherence, the central observable of interest.

% ------------------------------------------------------------------------------------------------

\subsection{On the Status of the Recursive Coupling Tensor}
\label{2.4.7:on_the_status_of_the_recursive_coupling_tensor}

The recursive coupling tensor \(R^\rho_{\mu\nu}\) possesses a dual nature:

\begin{enumerate}

    \item \textbf{As a Measurement:} It measures the coherence field's response to variations in the underlying semantic field:

    \begin{equation}
    \label{eq:R_measurement}
    R^\rho_{\mu\nu}(p, q, t) = \frac{\mathcal{D}^2 C^\rho(p,t)}{\mathcal{D} \psi^\mu(p) \mathcal{D} \psi^\nu(q)}
    \end{equation}

    \item \textbf{As a Dynamical Field:} It is an independent field whose evolution follows its own equation of motion, driven by the autopoietic potential:

    \begin{equation}
    \label{eq:R_dynamical}
    \frac{dR^\rho_{\mu\nu}(p,q,t)}{dt} = \Phi(C_{\mathrm{mag}}(p,t)) \cdot \chi^\rho_{\mu\nu}(p,q,t)
    \end{equation}

\end{enumerate}

This duality is resolved by a requirement that the time evolution of the measurement definition (Equation \ref{eq:R_measurement}) must equal the dynamical evolution equation (Equation \ref{eq:R_dynamical}). This imposes a formal constraint on the dynamics of the fundamental semantic field \(\psi^\mu\), such that the microscopic and macroscopic descriptions of the recursive coupling remain consistent at all times:

\begin{equation}
\label{eq:R_consistency}
\frac{d}{dt} \left( \frac{\mathcal{D}^2 C^\rho(p,t)}{\mathcal{D} \psi^\mu(p) \mathcal{D} \psi^\nu(q)} \right) = \Phi(C_{\mathrm{mag}}(p,t)) \cdot \chi^\rho_{\mu\nu}(p,q,t)
\end{equation}

To enforce the internal self-consistency of this theory, this equation is a condition the evolution of \(\psi^\mu\) and \(C^\rho\) must satisfy.

% ------------------------------------------------------------------------------------------------

\subsection{Scalar Measures from Vector Fields}
\label{2.4.8:scalar_measures_from_vector_fields}

Functions requiring scalar inputs derive them from vector fields using the metric. The primary example is the coherence magnitude:

\begin{equation}
C_{\mathrm{mag}}(p,t) = \sqrt{g_{\mu\nu}(p,t) C^\mu(p,t) C^\nu(p,t)}
\end{equation}

Potentials are functions of this scalar magnitude (e.g., \(V(C) := V(C_{\mathrm{mag}})\)). When a scalar potential influences vector dynamics, its gradient is taken with respect to the vector components via the chain rule; this preserves coordinate independence. 