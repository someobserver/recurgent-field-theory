\chapter{Field Index and Formal Architecture}
\label{2:field_index_and_formal_architecture}

% ------------------------------------------------------------------------------------------------

\section{Overview}
\label{2.1:overview}

Having established the axiomatic and conceptual groundwork, we now formalize these ideas with the full machinery of differential geometry. Here, each field and tensor embodies a distinct facet of the semantic landscape introduced in Chapter \ref{1:axiomatic_foundation}. The fields, tensors, and notations are drawn from differential geometry \autocite{Riemann1868, Lee2012}. We employ tensors because only they can naturally encode the kinds of nonlocal, multidimensional, and symmetry-rich relationships that recur in semantic structure.

% ------------------------------------------------------------------------------------------------

\section{Tensor Ranks and Properties}
\label{2.2:tensor_ranks_and_properties}

The framework we set forth is pseudo-Riemannian and constructed on an \(n\)-dimensional manifold \(\mathcal{M}\), referred to as the \textit{Semantic Manifold}. The tensor rank and symmetry properties of each field encode its geometric information, while its domain and range encode its semantic content. The metric tensor \(g_{\mu\nu}\) establishes the foundational structure (\S\ref{1.1:axiom_1_semantic_manifold}). The semantic and coherence fields, \(\psi^\mu\) and \(C^\mu\), provide the dynamic content (\S\ref{1.2:axiom_2_fundamental_semantic_field}), while third- and fourth-rank tensors mediate the feedback loops that drive manifold evolution (\S\ref{1.3:axiom_3_recursive_coupling}). All tensor expressions employ the Einstein summation convention, detailed in Section \ref{2.4:tensor_conventions_and_notation}.

To describe how \textit{meaning} and coherence unfold in the manifold, we need to formalize not only the topology of the space itself, but also the flows, forces, and regulatory principles acting within it. We thus partition the mathematical objects of RFT into six categories, each reflecting a different layer of semantic reality.

% ------------------------------------------------------------------------------------------------

\subsection{Fundamental Fields and Geometric Structure}
\label{2.2.1:fundamental_fields_and_geometric_structure}

{\small
\renewcommand{\arraystretch}{1.1}
\begin{longtable}{|c|p{5.5cm}|c|c|p{1.8cm}|c|c|}
\hline
\textbf{Symbol} & \textbf{Name} & \textbf{Rank} & \textbf{Symmetry} & \textbf{Domain} & \textbf{Range} & \textbf{Dim} \\
\hline
\endfirsthead
\hline
\textbf{Symbol} & \textbf{Name} & \textbf{Rank} & \textbf{Symmetry} & \textbf{Domain} & \textbf{Range} & \textbf{Dim} \\
\hline
\endhead
\(g_{\mu\nu}(p,t)\) & Metric tensor & (0,2) & Sym & \(\mathcal{M} \times \mathbb{R}\) & \(\mathbb{R}\) & \(n^2\) \\
\hline
\(\psi^\mu(p,t)\) & Semantic field & (1,0) & - & \(\mathcal{M} \times \mathbb{R}\) & \(T_p\mathcal{M}\) & \(n\) \\
\hline
\(C^\mu(p,t)\) & Coherence vector field & (1,0) & - & \(\mathcal{M} \times \mathbb{R}\) & \(T_p\mathcal{M}\) & \(n\) \\
\hline
\(\Gamma^\rho_{\mu\nu}\) & Christoffel symbols & (1,2) & Sym(\(\mu\),\(\nu\)) & \(\mathcal{M}\) & \(\mathbb{R}\) & \(n^3\) \\
\hline
\(v^\mu\) & Semantic velocity & (1,0) & - & \(\mathcal{M} \times \mathbb{R}\) & \(T_p\mathcal{M}\) & \(n\) \\
\hline
\caption{Fundamental Fields and Geometric Structure.}
\end{longtable}
}

% ------------------------------------------------------------------------------------------------

\subsection{Curvature and Geometric Quantities}
\label{2.2.2:curvature_and_geometric_quantities}

{\small
\renewcommand{\arraystretch}{1.1}
\begin{longtable}{|c|p{5.5cm}|c|c|p{1.8cm}|c|c|}
\hline
\textbf{Symbol} & \textbf{Name} & \textbf{Rank} & \textbf{Symmetry} & \textbf{Domain} & \textbf{Range} & \textbf{Dim} \\
\hline
\endfirsthead
\hline
\textbf{Symbol} & \textbf{Name} & \textbf{Rank} & \textbf{Symmetry} & \textbf{Domain} & \textbf{Range} & \textbf{Dim} \\
\hline
\endhead
\(R^{\rho}_{\sigma\mu\nu}\) & Riemann curvature tensor & (1,3) & Anti(\(\mu\),\(\nu\)) & \(\mathcal{M}\) & \(\mathbb{R}\) & \(n^4\) \\
\hline
\(R_{\mu\nu}\) & Ricci curvature tensor & (0,2) & Sym & \(\mathcal{M} \times \mathbb{R}\) & \(\mathbb{R}\) & \(n^2\) \\
\hline
\(G_{\rho\mu\nu}\) & Geometric structure tensor & (0,3) & Sym(\(\mu\),\(\nu\)) & - & \(\mathbb{R}\) & \(n^3\) \\
\hline
\(\kappa_t\) & Temporal curvature & 0 & - & \(\mathbb{R}^+\) & 1 & 1 \\
\hline
\caption{Curvature and Geometric Quantities}
\end{longtable}
}

% ------------------------------------------------------------------------------------------------

\subsection{Recursive Coupling and Feedback Dynamics}
\label{2.2.3:recursive_coupling_and_feedback_dynamics}

{\small
\renewcommand{\arraystretch}{1.1}
\begin{longtable}{|c|p{5.5cm}|c|c|p{1.8cm}|c|c|}
\hline
\textbf{Symbol} & \textbf{Name} & \textbf{Rank} & \textbf{Symmetry} & \textbf{Domain} & \textbf{Range} & \textbf{Dim} \\
\hline
\endfirsthead
\hline
\textbf{Symbol} & \textbf{Name} & \textbf{Rank} & \textbf{Symmetry} & \textbf{Domain} & \textbf{Range} & \textbf{Dim} \\
\hline
\endhead
\(R^\rho_{\mu\nu}(p,q,t)\) & Recursive coupling tensor & (1,2) & - & \(\mathcal{M}^2 \times \mathbb{R}\) & \(\mathbb{R}\) & \(n^3\) \\
\hline
\(\chi^\rho_{\mu\nu}(p,q,t)\) & Latent recursive channel tensor & (1,2) & - & \(\mathcal{M}^2 \times \mathbb{R}\) & \(\mathbb{R}\) & \(n^3\) \\
\hline
\(R^{\rho, \text{hetero}}_{\mu\nu}\) & Hetero-recursive tensor & (1,2) & - & \(\mathcal{M}^2 \times \mathbb{R}\) & \(\mathbb{R}\) & \(n^3\) \\
\hline
\(R^{(n)}\) & Meta-recursive tensor & (n,2n) & - & \(\mathcal{M}^n \times \mathbb{R}\) & \(\mathbb{R}\) & \(n^{3n}\) \\
\hline
\(S_{\mu\nu}(p,q)\) & Semantic similarity tensor\footnotemark[2] & (0,2) & Sym & \(\mathcal{M}^2\) & \(\mathbb{R}\) & \(n^2\) \\
\hline
\(H(p,q,t)\) & Historical co-activation\footnotemark[3] & 0 & - & \(\mathcal{M}^2 \times \mathbb{R}\) & \(\mathbb{R}^+\) & 1 \\
\hline
\(D^\rho_{\mu\nu}(p,q)\) & Domain incompatibility tensor & (1,2) & - & \(\mathcal{M}^2\) & \(\mathbb{R}^+\) & \(n^3\) \\
\hline
\(D(p,t)\) & Recursive depth & 0 & - & \(\mathcal{M} \times \mathbb{R}\) & \(\mathbb{N}\) & 1 \\
\hline
\caption{Recursive Coupling and Feedback Dynamics}
\end{longtable}
}

% ------------------------------------------------------------------------------------------------

\subsection{Physical Quantities and Dynamical Forces}
\label{2.2.4:physical_quantities_and_dynamical_forces}

{\small
\renewcommand{\arraystretch}{1.1}
\begin{longtable}{|c|p{5.5cm}|c|c|p{1.8cm}|c|c|}
\hline
\textbf{Symbol} & \textbf{Name} & \textbf{Rank} & \textbf{Symmetry} & \textbf{Domain} & \textbf{Range} & \textbf{Dim} \\
\hline
\endfirsthead
\hline
\textbf{Symbol} & \textbf{Name} & \textbf{Rank} & \textbf{Symmetry} & \textbf{Domain} & \textbf{Range} & \textbf{Dim} \\
\hline
\endhead
\(M(p,t)\) & Semantic mass & 0 & - & \(\mathcal{M} \times \mathbb{R}\) & \(\mathbb{R}^+\) & 1 \\
\hline
\(F_\mu(p,t)\) & Recursive force & (0,1) & - & \(\mathcal{M} \times \mathbb{R}\) & \(T_p^*\mathcal{M}\) & \(n\) \\
\hline
\(F_\mu^{\text{diss}}\) & Dissipative force & (0,1) & - & \(\partial\mathcal{A}\) & \(T_p^*\mathcal{M}\) & \(n\) \\
\hline
\(T_{\mu\nu}^{\text{rec}}\) & Recursive stress-energy tensor & (0,2) & Sym & \(\mathcal{M} \times \mathbb{R}\) & \(\mathbb{R}\) & \(n^2\) \\
\hline
\(P_{\mu\nu}\) & Recursive pressure tensor & (0,2) & Sym & \(\mathcal{M} \times \mathbb{R}\) & \(\mathbb{R}\) & \(n^2\) \\
\hline
\(F_{\mu\nu}\) & Metric forcing term & (0,2) & Sym & \(\mathcal{M}\) & \(\mathbb{R}\) & \(n^2\) \\
\hline
\(\rho(p,t)\) & Constraint density & 0 & - & \(\mathcal{M} \times \mathbb{R}\) & \(\mathbb{R}^+\) & 1 \\
\hline
\(\rho_V\) & Validation density & 0 & - & \(\mathcal{M} \times \mathbb{R}\) & \(\mathbb{R}^+\) & 1 \\
\hline
\caption{Physical Quantities and Dynamical Forces}
\end{longtable}
}

% ------------------------------------------------------------------------------------------------

\subsection{Potentials, Stability, and Phase Dynamics}
\label{2.2.5:potentials_stability_and_phase_dynamics}

{\small
\renewcommand{\arraystretch}{1.1}
\begin{longtable}{|c|p{5.5cm}|c|c|p{1.8cm}|c|c|}
\hline
\textbf{Symbol} & \textbf{Name} & \textbf{Rank} & \textbf{Symmetry} & \textbf{Domain} & \textbf{Range} & \textbf{Dim} \\
\hline
\endfirsthead
\hline
\textbf{Symbol} & \textbf{Name} & \textbf{Rank} & \textbf{Symmetry} & \textbf{Domain} & \textbf{Range} & \textbf{Dim} \\
\hline
\endhead
\(\Phi(C)\) & Autopoietic potential & 0 & - & \(T\mathcal{M}\) & \(\mathbb{R}^+\) & 1 \\
\hline
\(V(C)\) & Attractor potential\footnotemark[1] & 0 & - & \(T\mathcal{M}\) & \(\mathbb{R}^+\) & 1 \\
\hline
\(A(p,t)\) & Attractor stability\footnotemark[1] & 0 & - & \(\mathcal{M} \times \mathbb{R}\) & \([0,1]\) & 1 \\
\hline
\(\Theta(p,t)\) & Phase order parameter & 0 & - & \(\mathcal{M} \times \mathbb{R}\) & \(\mathbb{R}\) & 1 \\
\hline
\(W(p,t)\) & Wisdom field & 0 & - & \(\mathcal{M} \times \mathbb{R}\) & \(\mathbb{R}^+\) & 1 \\
\hline
\(\mathcal{H}[R]\) & Humility operator & 0 & - & \(\mathbb{R}\) & \(\mathbb{R}^+\) & 1 \\
\hline
\(S_R\) & Recurgence stability parameter & 0 & - & \(\mathcal{M} \times \mathbb{R}\) & \(\mathbb{R}^+\) & 1 \\
\hline
\(\mathcal{E}(t)\) & Recurgent expansion rate & 0 & - & \(\mathbb{R}\) & \(\mathbb{R}\) & 1 \\
\hline
\(S_{\text{sem}}\) & Semantic entropy & 0 & Func & \(P(\mathcal{M})\) & \(\mathbb{R}^+\) & 1 \\
\hline
\(\Gamma(\Omega)\) & Wisdom-coherence ratio & 0 & Func & \(P(\mathcal{M})\) & \(\mathbb{R}^+\) & 1 \\
\hline
\caption{Potentials, Stability, and Phase Dynamics}
\end{longtable}
}

% ------------------------------------------------------------------------------------------------

\subsection{Agent Fields and Communication Structures}
\label{2.2.6:agent_fields_and_communication_structures}

{\small
\renewcommand{\arraystretch}{1.1}
\begin{longtable}{|c|p{5.5cm}|c|c|p{1.8cm}|c|c|}
\hline
\textbf{Symbol} & \textbf{Name} & \textbf{Rank} & \textbf{Symmetry} & \textbf{Domain} & \textbf{Range} & \textbf{Dim} \\
\hline
\endfirsthead
\hline
\textbf{Symbol} & \textbf{Name} & \textbf{Rank} & \textbf{Symmetry} & \textbf{Domain} & \textbf{Range} & \textbf{Dim} \\
\hline
\endhead
\(\vec{P}^\mu\) & Proposition field & (1,0) & - & \(\mathcal{M} \times \mathbb{R}\) & \(T_p\mathcal{M}\) & \(n\) \\
\hline
\(\vec{V}_\mu\) & Validation field & (0,1) & - & \(\mathcal{M} \times \mathbb{R}\) & \(T_p^*\mathcal{M}\) & \(n\) \\
\hline
\(I^\mu\) & Interpretive field & (1,0) & - & \(\mathcal{M} \times \mathbb{R}\) & \(T_p\mathcal{M}\) & \(n\) \\
\hline
\(S_A\) & Agent attention field & 0 & - & \(\mathcal{M} \times \mathbb{R}\) & \([0,1]\) & 1 \\
\hline
\(N^\mu\) & Basis projection vector & (1,0) & - & - & \(T_p\mathcal{M}\) & \(n\) \\
\hline
\(T_{\mu\nu}^{(d \to d')}\) & Domain translation tensor & (0,2) & - & \(T\mathcal{M}_d \to T\mathcal{M}_{d'}\) & \(\mathbb{R}\) & \(n^2\) \\
\hline
\caption{Agent Fields and Communication Structures}
\end{longtable}
}

% ------------------------------------------------------------------------------------------------

\subsection{Operators, Functionals, and Constants}
\label{2.2.7:operators_functionals_and_constants}

{\small
\renewcommand{\arraystretch}{1.1}
\begin{longtable}{|c|p{5.5cm}|c|c|p{1.8cm}|c|c|}
\hline
\textbf{Symbol} & \textbf{Name} & \textbf{Rank} & \textbf{Symmetry} & \textbf{Domain} & \textbf{Range} & \textbf{Dim} \\
\hline
\endfirsthead
\hline
\textbf{Symbol} & \textbf{Name} & \textbf{Rank} & \textbf{Symmetry} & \textbf{Domain} & \textbf{Range} & \textbf{Dim} \\
\hline
\endhead
\(\Box\) & Covariant d'Alembertian & Op & \(C^2(\mathcal{M})\) & \(C^0(\mathcal{M})\) & - & - \\
\hline
\(\Delta_g\) & Laplace-Beltrami operator & Op & \(C^2(\mathcal{M})\) & \(C^0(\mathcal{M})\) & - & - \\
\hline
\(\nabla_f\) & Semantic forecast operator & Op & \(T\mathcal{M}\) & \(T\mathcal{M}\) & - & - \\
\hline
\(\mathcal{I}_{\psi}\) & Interpretive operator & Op & \(C^1(\mathcal{M})\) & \(C^1(\mathcal{M})\) & - & - \\
\hline
\(\mathcal{C}\) & Semantic compression operator & Op & \(P(\mathcal{M})\) & \(P(\mathcal{M}')\) & - & - \\
\hline
\(\mathcal{G}_\mu[\psi]\) & Recursive force functional & Func & \(C^1(\mathcal{M})\) & \(T_p^*\mathcal{M}\) & - & - \\
\hline
\hline
\multicolumn{7}{|c|}{\textbf{Physical Constants and Parameters}} \\
\hline
\(G_s\) & Semantic gravitational constant & 0 & - & \(\mathbb{R}^+\) & 1 & - \\
\hline
\(\gamma, \eta\) & Viscosity parameters & 0 & - & \(\mathbb{R}^+\) & 1 & - \\
\hline
\(k_V\) & Coherence rigidity & 0 & - & \(\mathbb{R}^+\) & 1 & - \\
\hline
\(\lambda_H\) & Humility strength & 0 & - & \(\mathbb{R}^+\) & 1 & - \\
\hline
\(\alpha_{\psi}\) & Microscopic coupling constant & 0 & - & \(\mathbb{R}\) & 1 & - \\
\hline
\(\alpha_{\Phi}\) & Autopoietic coupling constant & 0 & - & \(\mathbb{R}^+\) & 1 & - \\
\hline
\(\beta_{\Phi}\) & Critical exponent & 0 & - & \(\mathbb{R}^+\) & 1 & - \\
\hline
\(C_{\text{thr}}\) & Coherence threshold & 0 & - & \(\mathbb{R}^+\) & 1 & - \\
\hline
\(\hbar_s\) & Semantic uncertainty constant & 0 & - & \(\mathbb{R}^+\) & 1 & - \\
\hline
\caption{Operators, Functionals, and Constants}
\end{longtable}
}

\footnotetext[1]{The use of attractor stability metrics and potential energy landscapes for system characterization is drawn from nonlinear dynamics \autocite{Strogatz2014}.}

\footnotetext[2]{A formalization of the \textit{distributional hypothesis} in linguistics, which posits that words with similar distributions have similar meanings \autocite{Harris1954}. Other modern vector-space models of semantics, such as the Word2Vec framework, are built on this principle \autocite{Mikolov2013}.}

\footnotetext[3]{This serves as an implementation of Hebbian learning, which states that repeated, persistent co-activation of connected elements leads to an increase in the strength of their connection \autocite{Hebb1949}.}

% ------------------------------------------------------------------------------------------------

\section{System Architecture}
\label{2.3:system_architecture}

Coherence dynamics emerge from the interplay of four conceptual subsystems:

\begin{itemize}

    \item A geometric engine governs the evolution of the manifold's metric and curvature.

    \item A coherence processor handles the evolution of the primary fields.

    \item A recursive controller manages the coupling dynamics that link different regions of the manifold.

    \item A regulatory system provides wisdom and humility constraints.

\end{itemize}

The subsystems are deeply integrated to form two primary, coupled cycles. In the main causal loop, the coherence field determines a recursive stress-energy tensor, which in turn induces curvature in the metric. The deformed metric then governs the subsequent evolution of coherence, closing the primary feedback loop. 

Once coherence surpasses a critical threshold, a secondary generative cycle activates. This uses the autopoietic potential to form new recursive pathways, thereby driving genuine structural innovation. The entire system is modulated by the regulatory subsystem, which employs the wisdom field and humility operator, both emergent, to prevent pathological amplification and maintain dynamic equilibrium.

In sum, the framework rests on a interaction between geometric, field, recursive, and regulatory subarchitectures. Their dynamics, both cooperative and antagonistic, shape systemic capacity for order and novelty alike.

% ------------------------------------------------------------------------------------------------

\section{Tensor Conventions and Notation}
\label{2.4:tensor_conventions_and_notation}

The tensor conventions follow modern standards for differential geometry and tensor functions on smooth manifolds \autocite{Lee2012, MisnerThorneWheeler1973}. The framework from which this originates is the work of Gregorio Ricci Curbastro and Tullio Levi-Civita \autocite{RicciLeviCivita1901}.

Because semantic dynamics occur in a high-dimensional, non-Euclidean manifold, our notational conventions must encode both geometric and interpretive structure without ambiguity.

% ------------------------------------------------------------------------------------------------
\subsection{Index Notation and Einstein Summation}
\label{2.4.1:index_notation_and_einstein_summation}

We adopt the Einstein summation convention \autocite{Einstein1916}, under which any index appearing both as a subscript (covariant) and superscript (contravariant) within a single term is implicitly summed over all values from \(1\) to \(n\), where \(n\) denotes the manifold dimension. Greek indices \(\mu,\nu,\rho,\ldots\) range over manifold coordinates. For instance:

\begin{equation}
A_\mu B^\mu = \sum_{\mu=1}^n A_\mu B^\mu
\end{equation}

Index contraction—summing over paired covariant and contravariant indices—produces objects of reduced rank while encoding the fundamental tensor operation structure. This notation proves essential for expressing field interactions, geometric relationships, and derivatives across arbitrary coordinate charts on \(\mathcal{M}\). It provides us with manifestly covariant formulations of feedback dynamics and self-reference relationships throughout the semantic architecture.

% ------------------------------------------------------------------------------------------------
\subsection{Metric and Index Raising/Lowering}
\label{2.4.2:metric_and_index_raising_lowering}

The metric tensor \(g_{\mu\nu}(p,t)\) provides the geometric foundation for \(\mathcal{M}\), encoding distances, angles, and inner products within each tangent space \(T_p\mathcal{M}\). As semantic content and recursive structure dynamically curve the manifold, the metric evolves accordingly. The inverse metric \(g^{\mu\nu}(p,t)\) satisfies:

\begin{equation}
g_{\mu\rho}(p,t) g^{\rho\nu}(p,t) = \delta_\mu^\nu
\end{equation}

where \(\delta_\mu^\nu\) denotes the Kronecker delta.

Index raising and lowering operations \(C^\mu = g^{\mu\nu} C_\nu\) and \(C_\mu = g_{\mu\nu} C^\nu\) allow free interconversion between contravariant and covariant tensor components. This structural symmetry ensures proper transformation behavior under coordinate changes while maintaining covariant formulation of the field equations. The geometric relationships encoded in semantic space thus remain invariant regardless of manifold parameterization.

% ------------------------------------------------------------------------------------------------

\subsection{Covariant Derivatives}
\label{2.4.3:covariant_derivatives}

On a curved manifold, partial derivatives alone fail to respect the geometry of the space by not accounting for local effects induced by curvature. To differentiate tensor fields on \(\mathcal{M}\) in a way that preserves their geometric meaning under arbitrary coordinate transformations, we require the covariant derivative, denoted \(\nabla_\mu\).

For a covariant vector field \(C_\nu\), the derivative is defined as:

\begin{equation}
\nabla_\mu C_\nu = \partial_\mu C_\nu - \Gamma^\rho_{\mu\nu} C_\rho
\end{equation}

where \(\partial_\mu\) is the partial derivative with respect to the coordinate \(p^\mu\), and \(\Gamma^\rho_{\mu\nu}\) are the Christoffel symbols \autocite{Christoffel1869} of the second kind, given by:

\begin{equation}
\Gamma^\rho_{\mu\nu} = \frac{1}{2} g^{\rho\sigma} \left( \partial_\mu g_{\nu\sigma} + \partial_\nu g_{\mu\sigma} - \partial_\sigma g_{\mu\nu} \right)
\end{equation}

For contravariant vector fields \(V^\nu\), the covariant derivative takes the form:

\begin{equation}
\nabla_\mu V^\nu = \partial_\mu V^\nu + \Gamma^\nu_{\mu\rho} V^\rho
\end{equation}

In general, when differentiating a tensor of arbitrary rank, the covariant derivative includes a correction term involving the Christoffel symbols for each index, with a plus sign for contravariant indices and a minus sign for covariant indices.

Geometrically, the covariant derivative describes how a tensor field changes as it is "parallel transported" along the manifold, adjusting for the curvature encoded by \(g_{\mu\nu}\). In our context, semantic fields and their couplings evolve within a non-Euclidean, dynamically curved space, and any meaningful description of their dynamics or feedback relationships must remain invariant under coordinate transformations.

% ------------------------------------------------------------------------------------------------

\subsection{Functional and Variational Derivatives}
\label{2.4.4:functional_and_variational_derivatives}

The dynamical laws of Recurgent Field Theory are formulated via an action principle, following the tradition of modern field theories. The central object is the action functional, \(S\), defined as:

\begin{equation}
S = \int \mathcal{L}\left(C^\mu, \nabla_\nu C^\mu, g_{\mu\nu}, \ldots\right)\, dV,
\end{equation}

where \(\mathcal{L}\) is the Lagrangian density, a scalar function of the fields, their derivatives, the metric, and possibly other geometric quantities; and \(dV = \sqrt{|\det(g_{\mu\nu})|} \, d^n p\) is the invariant volume element on \(\mathcal{M}\). The principle of stationary action asserts that the physical evolution of the system extremizes \(S\) with respect to variations in the field configurations. That is, the true trajectory of the system is one for which the first variation of the action, \(\delta S = 0\), for arbitrary variations \(\delta C^\mu(p)\) that vanish at the boundary.

This leads, via the calculus of variations, to the Euler-Lagrange equations for tensor fields. For a field \(C^\mu\), the equation of motion reads:

\begin{equation}
\frac{\delta \mathcal{L}}{\delta C^\mu} = \frac{\partial \mathcal{L}}{\partial C^\mu} - \nabla_\nu \left( \frac{\partial \mathcal{L}}{\partial (\nabla_\nu C^\mu)} \right) = 0
\end{equation}

Here, \(\frac{\delta \mathcal{L}}{\delta C^\mu}\) denotes the variational (or functional) derivative of the Lagrangian density with respect to the field. The variational derivative generalizes ordinary differentiation to the infinite-dimensional space of field configurations, encoding how infinitesimal changes in the field \(C^\mu(p)\) at each point \(p \in \mathcal{M}\) affect the action as a whole.

This approach has several conceptual and practical benefits. It allows for the systematic derivation of field equations governing the dynamics of semantic coherence, recursive coupling, and geometric structure, while keeping the equations covariant. As a concrete example, for any Lagrangian density depending on \(C^\mu\) and its covariant derivative \(\nabla_\nu C^\mu\), the corresponding Euler-Lagrange equation takes the explicit form:

\begin{equation}
\frac{\partial \mathcal{L}}{\partial C^\mu} - \nabla_\nu \left( \frac{\partial \mathcal{L}}{\partial (\nabla_\nu C^\mu)} \right) = 0
\end{equation}

% ------------------------------------------------------------------------------------------------

\subsection{Integration and Symmetries}
\label{2.4.5:integration_and_symmetries}

All integrals over the Semantic Manifold are performed using the invariant volume element, \(dV\), which is defined as:

\begin{equation}
dV = \sqrt{|\det(g_{\mu\nu})|} \, d^n p,
\end{equation}

where \(d^n p\) is the Lebesgue measure in local coordinates and \(\det(g_{\mu\nu})\) is the determinant of the metric tensor at each point. This construction guarantees integrals of scalar quantities remain unchanged under arbitrary smooth reparameterizations of the manifold. Tensor symmetries are likewise central to both the mathematical efficiency and conceptual coherence of Recurgent Field Theory.

When present, index symmetries (such as \(g_{\mu\nu} = g_{\nu\mu}\) for the metric or anti-symmetry in the Riemann tensor) are explicitly noted and exploited throughout, both to reduce computational complexity and to reveal underlying conservation laws. For example, the total “semantic mass” or any other scalar observable \(f(p)\) can be meaningfully defined as:

\begin{equation}
\mathcal{M}_{\text{total}} = \int_\mathcal{M} f(p) \, dV,
\end{equation}

with the guarantee that its value depends only on the field content, not on the coordinate chart used.


% ------------------------------------------------------------------------------------------------

\subsection{Fundamental versus Derived Fields}
\label{2.4.6:fundamental_versus_derived_fields}

We distinguish between the fundamental dynamical variables and the quantities derived from them. The \textbf{semantic field}, \(\psi^\mu(p,t)\), represents the raw, underlying semantic content at each point of the manifold. It is the primary dynamical variable of the theory, encoding pure potentiality.

The \textbf{coherence field}, \(C^\mu(p,t)\), is a derived object measuring the degree of self-consistency and alignment present in the semantic field. As an \(n\)-dimensional vector field, each component quantifies coherence along a principal semantic axis. Importantly, \(C^\mu\) is constructed as a nonlocal functional of \(\psi^\mu\):

\begin{equation}
C^\mu(p,t) = \mathcal{F}^\mu[\psi](p,t) = \int_{\mathcal{N}(p)} K^\mu_{\ \nu}(p,q) \psi^\nu(q,t) \, dq
\end{equation}

where \(K^\mu_{\ \nu}(p,q)\) is a kernel that may encode locality, similarity, or coupling between points.

The theory is built such that only coherent, organized semantic content, rather than mere potential, drives system dynamics and observable structure. Accordingly, while the full dynamics can be expressed in terms of \(\psi^\mu\), the Lagrangian is formulated using \(C^\mu\) to maintain direct connection to the system's measurable, interpretable coherence.

In physical analogy, this is akin to distinguishing between a quantum wavefunction (potentiality) and a probability density (actualization), or between a configuration field and an observable order parameter in statistical physics.

% ------------------------------------------------------------------------------------------------

\subsection{On the Status of the Recursive Coupling Tensor}
\label{2.4.7:on_the_status_of_the_recursive_coupling_tensor}

The recursive coupling tensor \(R^\rho_{\mu\nu}\) occupies a pivotal and dual role within this theory. It serves simultaneously as a measure of system sensitivity, and as a dynamical field in its own right, with its own evolution.

\textbf{As a Measurement.} First, \(R^\rho_{\mu\nu}\) quantifies how the coherence field \(C^\rho\) at point \(p\) and time \(t\) responds to infinitesimal variations in the underlying semantic field \(\psi^\mu\) at \(p\) and \(\psi^\nu\) at \(q\):

\begin{equation}
\label{eq:R_measurement}
R^\rho_{\mu\nu}(p, q, t) = \frac{\mathcal{D}^2 C^\rho(p, t)}{\mathcal{D} \psi^\mu(p) \mathcal{D} \psi^\nu(q)}
\end{equation}

This second variational derivative captures the nonlocal influence of semantic fluctuations on system-level coherence. It functions as a generalized curvature in the space of fields, encoding how semantic structure bends or couples recursively.

\textbf{As a Dynamical Field.} Second, \(R^\rho_{\mu\nu}\) appears as an independent, time-evolving field. Its dynamics are governed by the interplay of autopoietic potential and latent recursive channels:

\begin{equation}
\label{eq:R_dynamical}
\frac{dR^\rho_{\mu\nu}(p, q, t)}{dt} = \Phi(C_{\mathrm{mag}}(p, t)) \cdot \chi^\rho_{\mu\nu}(p, q, t)
\end{equation}

Here, \(\Phi\) is the autopoietic potential, a scalar function reflecting self-organizing tendencies, and \(\chi^\rho_{\mu\nu}\) is a latent channel tensor encoding "hidden" recursive capacities.

\textbf{Consistency Constraint.} For theoretical coherence, the measurement-based and dynamical definitions of \(R^\rho_{\mu\nu}\) must coincide in their time evolution. This imposes a formal constraint:

\begin{equation}
\label{eq:R_consistency}
\frac{d}{dt} \left( \frac{\mathcal{D}^2 C^\rho(p, t)}{\mathcal{D} \psi^\mu(p) \mathcal{D} \psi^\nu(q)} \right) = \Phi(C_{\mathrm{mag}}(p, t)) \cdot \chi^\rho_{\mu\nu}(p, q, t)
\end{equation}

This equation demands that the dynamical evolution of the fundamental semantic field \(\psi^\mu\) and the coherence field \(C^\rho\) is constrained such that the microscopic (variational) and macroscopic (dynamical) accounts of recursive coupling remain consistent at all times.

Self-consistency is the mechanism by which this theory ties together feedback, self-reference, and the emergence of higher-order structure. The constraint acts as a kind of "semantic Bianchi identity," maintaining the internal logical closure of RFT's recursion dynamics.

% ------------------------------------------------------------------------------------------------

\subsection{Scalar Measures from Vector Fields}
\label{2.4.8:scalar_measures_from_vector_fields}

Potentials, order parameters, stability criteria, and other parameters of interest require scalar inputs derived from underlying vector or tensor fields. This reduction is performed using the metric, which provides a coordinate-invariant way to extract scalar magnitudes. The canonical example is the magnitude of the coherence field:

\begin{equation}
C_{\mathrm{mag}}(p,t) = \sqrt{g_{\mu\nu}(p,t) C^\mu(p,t) C^\nu(p,t)}
\end{equation}

This construction ensures that quantities like coherence magnitude remain independent of local coordinates, relying solely on the intrinsic geometry of \(\mathcal{M}\).

Potentials and other scalar-valued functions (such as the autopoietic potential \(V(C)\)) are then defined in terms of \(C_{\mathrm{mag}}\). When they influence the evolution of vector fields, their gradients are computed with respect to the vector components, applying the chain rule as appropriate. This preserves the coordinate independence and covariant structure of the theory, maintaining consistency with the overall geometric framework.

As a result, all scalar diagnostics, potentials, and stability measures inherit the symmetry and invariance properties of the underlying manifold, allowing the theory to express global system properties in a form accessible to both calculation and interpretation.