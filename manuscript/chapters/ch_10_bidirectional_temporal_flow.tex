\chapter{Bidirectional Temporal Flow}

\section{Overview}

In semantic systems, time cannot work as a one-way street. Powerful ideas reach backward, reinterpreting the past and reshaping the very context from which they emerged. Suddenly learning a new route to work can change how you interpret your memory of living in the city. The same process generating new meaning can simultaneously recontextualize the field from which it arose. Or better stated: updated information in the present can affect the meaning of past events just the same way present decisions can affect all future interpretations and decisions.

This chapter formalizes bidirectional temporal flow. It proposes causal influence propagating forward from concentrations of semantic mass with informational receptivity flowing backward, guided by wisdom and constraint. The concept of backpropagation of influence finds its parallels in views on the nature of quantum mechanics, such as Wheeler's participatory universe and Cramer's transactional interpretation \autocite{WheelerZurek1983, Cramer1986}. The following defines the dual vector fields governing asymmetric temporality, giving rise to phenomena like perceived temporal curvature near major semantic attractors.

\section{Formal Structure}

Let \(\mathcal{M}\) denote the semantic manifold. Define two vector fields:

\begin{enumerate}
    \item Causal Emission Field \(\vec{E}_c(p,t)\):
    \begin{equation}
    \vec{E}_c(p,t) = \gamma_c\, M(p,t)\, \nabla\Phi(p,t)
    \end{equation}
    where
    \begin{itemize}
        \item \(\gamma_c\) is the causal coupling constant,
        \item \(M(p,t)\) is the semantic mass density,
        \item \(\nabla\Phi(p,t)\) is the gradient of the recursive potential.
    \end{itemize}

    \item Information Reception Field \(\vec{I}_r(p,t)\):
    \begin{equation}
    \vec{I}_r(p,t) = -\gamma_i\, \rho(p,t)\, \nabla W(p,t)
    \end{equation}
    where
    \begin{itemize}
        \item \(\gamma_i\) is the information coupling constant,
        \item \(\rho(p,t)\) is the constraint density,
        \item \(\nabla W(p,t)\) is the gradient of the wisdom field.
    \end{itemize}
\end{enumerate}

The interaction of these fields is quantified by the temporal asymmetry operator:

\begin{equation}
\mathcal{T}(p,t) = \vec{E}_c(p,t) \cdot \vec{I}_r(p,t)
\end{equation}

which measures the local alignment between causal emission and information reception at each point \((p,t) \in \mathcal{M}\).

\subsection{Conservation Principle}

The bidirectional temporal flow is governed by a conservation law:

\begin{equation}
\nabla \cdot \vec{E}_c(p,t) + \frac{\partial}{\partial t} I_d(p,t) = 0
\end{equation}

where \(I_d(p,t) = \|\vec{I}_r(p,t)\|\) denotes the information density.

The relation asserts the divergence of causal emission is balanced by the negative temporal rate of change of information density. This is a semantic analogue to the continuity equations of classical field theory \autocite{Ryder1996, PeskinSchroeder1995}.

\subsection{Recursive Temporal Curvature}

In regions of elevated semantic mass, such as at recursive attractors, the interplay of bidirectional flows gives rise to a recursive temporal lens effect, formalized by the temporal curvature coefficient:

\begin{equation}
\kappa_t(p) = \frac{\|\vec{E}_c(p,t)\|}{\|\vec{I}_r(p,t)\|} \cdot \frac{1}{1 + \lambda\, \|R_{ijk}(p,p,t)\|_F}
\end{equation}

where
\begin{itemize}
    \item \(\kappa_t(p)\) quantifies the local curvature of temporal flow,
    \item \(\lambda\) is a damping parameter,
    \item \(\|R_{ijk}(p,p,t)\|_F\) is the Frobenius norm of the self-recursive coupling tensor.
\end{itemize}

Interpretation:
\begin{itemize}
    \item \(\kappa_t(p) \gg 1\) indicates dominance of causal emission, corresponding to temporal dilation.
    \item \(\kappa_t(p) \ll 1\) indicates dominance of information reception, corresponding to temporal contraction.
\end{itemize}

\subsection{Modification of Coherence Dynamics}

The bidirectional temporal structure remains well-defined for arbitrary manifold dimension and structure. This modifies the evolution of the coherence field \(C_i(p,t)\) as follows:

\begin{equation}
\frac{\partial C_i(p,t)}{\partial t} = \Box C_i + T_{ij}^{\mathrm{rec}}\, g^{jk} C_k + \xi\, (\vec{E}_c \times \vec{I}_r)_i
\end{equation}

where \(\Box\) is the d'Alembertian (or appropriate Laplacian) operator, \(T_{ij}^{\mathrm{rec}}\) encodes recursive coupling, \(g^{jk}\) is the inverse metric, and \(\xi\) is a coupling constant.

The cross product term \((\vec{E}_c \times \vec{I}_r)_i\) is defined according to the dimension \(n\) of the manifold:
\begin{itemize}
    \item For \(n=3\), the standard vector cross product applies.
    \item For \(n \neq 3\), employ the antisymmetric tensor:
    \begin{equation}
    (\vec{E}_c \times \vec{I}_r)_i = \omega_{ijk} E_c^j I_r^k
    \end{equation}
    where \(\omega_{ijk}\) is the orientation tensor.
    \item In a fully coordinate-free, dimension-agnostic formulation:
    \begin{equation}
    (\vec{E}_c \times \vec{I}_r)_i = \left(\star(E_c^\flat \wedge I_r^\flat)\right)^\sharp_i
    \end{equation}
    where \(\star\) is the Hodge star, \(\wedge\) the exterior product, and \(\flat\), \(\sharp\) denote musical isomorphisms.
\end{itemize} 