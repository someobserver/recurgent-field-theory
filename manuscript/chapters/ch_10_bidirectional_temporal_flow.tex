\chapter{Bidirectional Temporal Flow}

\section{Overview}

Classical physics treats time as a unidirectional parameter. In semantic systems, however, the "arrow of time" is more complex. The discovery of a new truth can reach backward to reshape an observer's interpretation of past events, just as a present decision shapes the future. This phenomenon is formalized here through the interaction of forward and backward-propagating fields, inspired by the transactional interpretation of quantum mechanics \autocite{Cramer1986}. A "proposition" about meaning projects from the past and receives "validation" from a future state of high wisdom.

\section{Forward and Backward-Propagating Potentials}

This model requires two vector fields on the manifold.

\subsection{The Proposition Field}
The Proposition field, \(\vec{P}(p,t)\), represents the "proposition" that a semantic structure makes to the future. Concentrations of semantic mass source this forward-propagating potential. Its strength is proportional to the structure's mass and propagation velocity.
\begin{equation}
\vec{P}(p,t) = \gamma_p M(p,t) \vec{v}(p,t)
\end{equation}
where \(M\) is the semantic mass, \(\vec{v}\) is the semantic velocity field (\(\partial\psi/\partial t\)), and \(\gamma_p\) is a coupling constant. This field represents the causal push of an existing meaning proposing itself for future relevance.

\subsection{The Validation Field}
The Validation field, \(\vec{V}(p,t)\), represents the "validation" sent back from a future state. Gradients in the wisdom field source this backward-propagating potential. This represents the interpretive pull from regions of anticipated understanding.
\begin{equation}
\vec{V}(p,t) = -\gamma_v \nabla W(p,t)
\end{equation}
where \(\nabla W\) is the gradient of the wisdom field and \(\gamma_v\) is a coupling constant. The field flows "down" the wisdom gradient toward regions of higher wisdom, selecting and confirming viable propositions.

\section{Temporal Interaction in the Lagrangian}

The transaction between a proposition and its validation is integral to the system's energetics. A new scalar interaction term, \(\mathcal{L}_{\text{temporal}}\), introduced into the system Lagrangian (Chapter 6) models this transaction.
\begin{equation}
\mathcal{L}_{\text{total}} = \mathcal{L}_{\text{RFT}} + \mathcal{L}_{\text{temporal}}
\end{equation}
The interaction term is defined by the covariant inner product of the two fields:
\begin{equation}
\mathcal{L}_{\text{temporal}} = \xi \, g^{ij} P_{i} V_{j}
\end{equation}
where \(\xi\) is the temporal coupling constant. A completed transaction contributes positively to the action, making such paths more probable through strong alignment between a proposition and a validation.

\section{Modified Field Dynamics and Consequences}

The introduction of \(\mathcal{L}_{\text{temporal}}\) modifies the equations of motion. The variational principle (\(\delta S = 0\)), applied to the new total Lagrangian, adds a new force term, \(\vec{F}_{\text{temporal}}\), to the Euler-Lagrange equation for the coherence field:
\begin{equation}
\Box C^i + \dots + \lambda \frac{\partial \mathcal{H}[R]}{\partial C_i} - F^i_{\text{temporal}} = 0
\end{equation}
where \(F^i_{\text{temporal}} = \delta(\int \mathcal{L}_{\text{temporal}} dV) / \delta C_i\). This term introduces the influence of the bidirectional temporal flow into the coherence dynamics.

\subsection{Conservation and Temporal Curvature}

The flow of propositions and validations is balanced and preserved by the conservation principle through the continuity equation:
\begin{equation}
\nabla_i P^i + \frac{\partial \rho_V}{\partial t} = 0
\end{equation}
where \(\rho_V = \sqrt{g^{ij} V_{i} V_{j}}\) is the scalar validation density. The divergence of the forward-propagating proposition field is balanced by the change in density of the backward-propagating validation field.

The relative strength of these two fields at a point defines the local temporal curvature, \(\kappa_t\), which measures the perceived rate of temporal flow near a semantic structure.
\begin{equation}
\kappa_t(p) = \frac{\|\vec{P}(p)\|}{\|\vec{V}(p)\|}
\end{equation}
When \(\kappa_t \gg 1\), the causal "push" of propositions dominates, producing a subjective sense of temporal dilation. When \(\kappa_t \ll 1\), the "pull" of a future validation dominates, producing a sense of temporal contraction as the system rapidly reconfigures toward a new understanding. 