\chapter{Recurgent Field Equation and Lagrangian Mechanics}

\section{Overview}

Semantic structures and their evolution are governed by the principle of stationary action \autocite{GoldsteinPooleSafko2002, Arnold1989}. The same mathematical machinery allows the complexity of competing semantic forces to be encoded in a single object: the Lagrangian. We derive the fundamental equations of motion for the semantic field by finding the path that minimizes the action. This chapter details the RFT Lagrangian and the resulting Euler-Lagrange field equation, describing the manner in which coherence propagates and evolves across the manifold.

\section{Lagrangian Density}

Semantic dynamics involve competing forces: the drive toward coherence, autopoietic generative potential, and regulatory constraints preventing runaway recursion. The Lagrangian encodes their influences while keeping the resulting field equations consistent with fundamental symmetries and conservation laws \autocite{Lagrange1788, Euler1744, LandauLifshitz1975}.

The Lagrangian density captures the energetic landscape of semantic evolution with terms representing (1) kinetic flow, (2) attractor dynamics, (3) generative potential, and (4) regulatory constraints:

\begin{equation}
\mathcal{L} = \mathcal{L}_{\text{kinetic}} - V(C_{\text{mag}}) + \Phi(C_{\text{mag}}) - \lambda \mathcal{H}[R]
\end{equation}

The first term quantifies the kinetic energy of coherence gradients (the cost of non-uniform coherence across the manifold). The potential terms encode fundamental tensions driving semantic evolution: \(V(C_{\text{mag}})\) represents the stabilizing influence of semantic attractors, \(\Phi(C_{\text{mag}})\) describes the generative capacity for autopoietic innovation, and \(\mathcal{H}[R]\) provides regulatory humility constraint preventing excessive recursive amplification.

Remark on Real and Complex Coherence Fields: The Lagrangian above is formulated for a real coherence field \(C_i\). For systems exhibiting phase dynamics, a complexified Lagrangian is employed:
\begin{equation}
\mathcal{L}_C = \frac{1}{2} g^{ij} (\nabla_i C_k)(\nabla_j C^{k*}) - V(C_{\mathrm{mag}}) + \Phi(C_{\mathrm{mag}}) - \lambda \cdot \mathcal{H}[R]
\end{equation}
where \(C^{k*}\) denotes the complex conjugate of \(C^k\) and \(C_{\mathrm{mag}} = \sqrt{g^{ij} C_i C_j^*}\). This extension is required for describing wave-like and phase-dependent recurgent phenomena.

\section{Action Principle}

The action functional is given by

\begin{equation}
S = \int_{\mathcal{M}} \mathcal{L} \, dV
\end{equation}

The system's dynamics follow from the principle of stationary action: physical evolution corresponds to stationary points of \(S\) under admissible variations, subject to imposed constraints.

\section{Euler–Lagrange Field Equation}

Variation of the action with respect to \(C_i\) yields the Euler–Lagrange equation \autocite{Euler1744, Lagrange1788}:

\begin{equation}
\frac{\delta \mathcal{L}}{\delta C_i} - \nabla_j \left( \frac{\delta \mathcal{L}}{\delta (\nabla_j C_i)} \right) = 0
\end{equation}

which, for the Lagrangian above, takes the explicit form

\begin{equation}
\Box C^i + \frac{\partial V(C_{\mathrm{mag}})}{\partial C_i} - \frac{\partial \Phi(C_{\mathrm{mag}})}{\partial C_i} + \lambda \cdot \frac{\partial \mathcal{H}[R]}{\partial C_i} = 0
\end{equation}

where

\begin{itemize}
    \item \(\Box = \nabla^a \nabla_a\) is the covariant d'Alembertian (semantic Laplacian).
\end{itemize}

The derivatives of the scalar potentials with respect to the vector field components are computed via the chain rule:

\begin{equation}
\frac{\partial V(C_{\mathrm{mag}})}{\partial C_i} = \frac{dV}{dC_{\mathrm{mag}}} \cdot \frac{\partial C_{\mathrm{mag}}}{\partial C_i} = \frac{dV}{dC_{\mathrm{mag}}} \cdot \frac{g^{ij} C_j}{C_{\mathrm{mag}}}
\end{equation}

\begin{equation}
\frac{\partial \Phi(C_{\mathrm{mag}})}{\partial C_i} = \frac{d\Phi}{dC_{\mathrm{mag}}} \cdot \frac{\partial C_{\mathrm{mag}}}{\partial C_i} = \frac{d\Phi}{dC_{\mathrm{mag}}} \cdot \frac{g^{ij} C_j}{C_{\mathrm{mag}}}
\end{equation}

The humility constraint term involves a more intricate dependence, as \(R\) is a functional of \(C\) via the underlying semantic field \(\psi\):

\begin{equation}
\frac{\partial \mathcal{H}[R]}{\partial C_i} = \int_{\mathcal{M}} \frac{\delta \mathcal{H}[R]}{\delta R_{jkl}(s, t)} \cdot \frac{\delta R_{jkl}(s, t)}{\delta C_i(p)} \, dV_s
\end{equation}

The final term thus encodes the indirect coupling between \(C_i\) and \(R_{jkl}\), mediated by \(\psi\).

Given the evolution equation for \(R\),
\begin{equation}
\frac{dR_{ijk}}{dt} = \Phi(C_{\mathrm{mag}}) \cdot \chi_{ijk},
\end{equation}
the humility constraint \(\mathcal{H}[R]\) introduces a nontrivial feedback mechanism, whereby the present state of coherence modulates the future structure of recursive coupling.

\section{Structural Interpretation}

The above formalism constitutes a semantic field theory structurally analogous to established physical field theories such as general relativity \autocite{Einstein1915, Wald1984} and Yang-Mills theory \autocite{PeskinSchroeder1995}:

\begin{itemize}
    \item The curvature term (\(\Box\)) governs the propagation of recursive structure,
    \item The potentials (\(V(C)\), \(\Phi(C)\)) define the landscape of stable and generative attractors,
    \item The constraint (\(\mathcal{H}\)) regulates recursion.
\end{itemize}

The resulting theory describes the evolution of coherence under the combined influence of gradient flow, potential-driven dynamics, and constraint enforcement.

\section{Coupled Field Dynamics}

Although the Lagrangian and resulting field equations are expressed in terms of the coherence field \(C_i\), a complete description requires explicit consideration of the underlying semantic field \(\psi_i\) and its evolution.

\subsection{Semantic Field Evolution}

The semantic field \(\psi_i\) evolves according to

\begin{equation}
\frac{\partial \psi_i(p, t)}{\partial t} = v_i(p, t)
\end{equation}

where the semantic velocity field \(v_i(p, t)\) is given by

\begin{equation}
v_i(p, t) = \alpha \cdot \nabla_i C_{\mathrm{mag}}(p, t) + \beta \cdot F_i(p, t) + \gamma \cdot \mathcal{R}_i[\psi](p, t)
\end{equation}

with

\begin{itemize}
    \item \(\alpha \cdot \nabla_i C_{\mathrm{mag}}(p, t)\): Gradient-driven flow toward regions of higher coherence,
    \item \(\beta \cdot F_i(p, t)\): Recursive force arising from the surrounding semantic mass,
    \item \(\gamma \cdot \mathcal{R}_i[\psi](p, t)\): Direct recursive feedback.
\end{itemize}

This establishes a bidirectional coupling:

\begin{enumerate}
    \item \(\psi_i\) determines \(C_i\) via the coherence functional,
    \item \(C_i\) influences the evolution of \(\psi_i\) through its gradient.
\end{enumerate}

\subsection{Full Dynamical System}

The coupled system is thus:

\begin{equation}
\frac{\partial \psi_i(p, t)}{\partial t} = v_i(p, t)
\end{equation}

\begin{equation}
\Box C_i + \frac{\partial V}{\partial C_i} - \frac{\partial \Phi}{\partial C_i} + \lambda \cdot \frac{\partial \mathcal{H}}{\partial C_i} = 0
\end{equation}

\begin{equation}
C_i(p, t) = \mathcal{F}_i[\psi](p, t)
\end{equation}

This system may be integrated numerically by updating \(\psi_i\) and deriving \(C_i\) at each time step, or, in certain analytical regimes, reformulated to eliminate \(\psi_i\) in favor of a closed evolution for \(C_i\).

\subsection{Consistency of the Action Principle}

For the variational structure to hold, variations in \(C_i\) must correspond to admissible variations in \(\psi_i\). This is formalized via constrained variation:

\begin{equation}
\delta C_i(p, t) = \int_{\mathcal{M}} \frac{\delta C_i(p, t)}{\delta \psi_j(q, t)} \, \delta \psi_j(q, t) \, dq
\end{equation}

The action principle continues to apply when such constraints are incorporated, so the coupled evolution of \(C_i\) and \(\psi_i\) remains compatible with variational structure.