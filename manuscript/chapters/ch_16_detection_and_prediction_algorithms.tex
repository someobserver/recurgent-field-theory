\chapter{Detection and Prediction Algorithms}

\section{Overview}

This chapter establishes the computational bridge between the abstract theory and its practical application. The goal is an algorithm able to analyze semantic field data, identify the geometric signatures of the pathologies from Chapter 15, and forecast their evolution. This requires discretizing the continuous manifold \(\mathcal{M}\) and its associated fields, and solving the core differential equations with stable numerical methods. The methods used are chosen for their robustness and proven convergence properties and are standard within the theory of computation \autocite{Sipser2012}.

\section{Algorithmic Foundation}

\subsection{Semantic Manifold Discretization}

A discrete set of points, or a lattice, represents the continuous semantic manifold \(\mathcal{M}\), where each point \(p_i\) holds a vector of field values.
\begin{equation}
p_i(t) = \{\psi_i(t), C_i(t), g_{ij}(t), M_i(t), W_i(t)\}
\end{equation}
The components are the core fields of the theory: the fundamental semantic field \(\psi\), coherence field \(C\), metric \(g_{ij}\), semantic mass \(M\), and wisdom field \(W\). The reference implementation represents the fields \(\psi\) and \(C\) as 2000-dimensional vectors.

\subsection{Metric and Curvature Tensors}

The metric tensor \(g_{ij}\) is fundamental; it defines the geometry from which all other properties derive. It is computed from the semantic field's gradients with a second-order finite difference approximation, a standard technique in numerical analysis \autocite{BurdenFairesBurden2015}.
\begin{equation}
g_{ij}(p,t) = \sum_{k=1}^n \frac{\partial \psi_k}{\partial x^i} \frac{\partial \psi_k}{\partial x^j} + \delta_{ij}, \quad \text{where} \quad \frac{\partial \psi_k}{\partial x^i} \approx \frac{\psi_k(x + h e_i) - \psi_k(x - h e_i)}{2h}
\end{equation}
The Christoffel symbols \(\Gamma^k_{ij}\) and the full Riemann curvature tensor \(R^{\rho}_{\sigma\mu\nu}\) are then computed from the discretized metric field via their standard definitions, using finite differences for the required derivatives. These tensors are the direct geometric indicators of pathological curvature.

\subsection{Recursive Coupling Tensor}

The recursive coupling tensor \(R_{ijk}\) has a theoretical definition as a second derivative. Its numerical implementation must accurately reflect this. A direct, second-order finite difference approximation replaces the previous heuristic:
\begin{equation}
R_{ijk}(p,q,t) = \frac{\partial^2 C_k(p,t)}{\partial \psi_i(p) \partial \psi_j(q)} \approx \frac{C_k(p)_{\psi_i^+,\psi_j^+} - C_k(p)_{\psi_i^+,\psi_j^-} - C_k(p)_{\psi_i^-,\psi_j^+} + C_k(p)_{\psi_i^-,\psi_j^-}}{4h_i h_j}
\end{equation}
where \(C_k(p)_{\psi_i^+,\psi_j^+}\) denotes the coherence field at \(p\) evaluated with a positive perturbation of magnitude \(h_i\) to \(\psi_i\) at \(p\) and a positive perturbation of magnitude \(h_j\) to \(\psi_j\) at \(q\). This rigorous formulation accurately models the subtle dynamics of recursive influence.

\section{Dynamical Evolution and Analysis}

\subsection{Geodesics and Field Trajectories}

Solving the geodesic equation traces the paths of semantic concepts, which identifies, for instance, when a pathological attractor captures a thought process.
\begin{equation}
\frac{d^2 x^{\mu}}{d\tau^2} + \Gamma^{\mu}_{\alpha\beta} \frac{dx^{\alpha}}{d\tau} \frac{dx^{\beta}}{d\tau} = 0
\end{equation}
A fourth-order Runge-Kutta integrator, a classic method for accuracy and stability, solves this system of ordinary differential equations \autocite{Runge1895, Kutta1901}. The same method, with implicit time-stepping for the nonlinear recursive term, applies to the main field evolution equation, \(\Box C + T^{\text{rec}}[\partial C] = 0\).

\subsection{Stability Analysis via Lyapunov Exponents}

The maximal Lyapunov exponent, \(\lambda_{\max}\), introduced in Lyapunov's seminal work on the stability of dynamical systems and later generalized by the multiplicative ergodic theorem \autocite{Lyapunov1907, Oseledets1968}, determines if a semantic region is stable, chaotic, or pathologically rigid. It quantifies the divergence rate of nearby trajectories in phase space. A positive \(\lambda_{\max}\) is a hallmark of chaos (often seen in Fragmentation pathologies), while \(\lambda_{\max} \approx 0\) can indicate the rigidity of Belief Calcification.
\begin{equation}
\lambda_{\max} = \lim_{t \to \infty} \frac{1}{t} \ln \frac{\|\delta C(t)\|}{\|\delta C(0)\|}
\end{equation}
The calculation requires integrating the linearized equations of motion for a perturbation vector \(\delta C\) alongside the main field evolution.

\subsection{Spectral Analysis of Geometric Operators}

The spectral properties of a semantic structure's geometric operators reveal its underlying "resonant frequencies." The eigenvalues of the Laplace-Beltrami operator, \(\Delta_g\), are computed; its spectrum encodes the manifold's intrinsic scale and connectivity, analogous to the vibrational modes of a drumhead \autocite{Chung1997}.
\begin{equation}
\Delta_g \phi_n = \lambda_n \phi_n
\end{equation}
A sparse spectrum with a large gap after the first few eigenvalues indicates a well-structured, coherent manifold, while a dense, continuous spectrum suggests the disorganization of a Fragmentation pathology.

\subsection{Topological Data Analysis}

Beyond spectral methods, the tools of computational topology offer a way to quantify the shape of the semantic manifold. Persistent homology, a technique in topological data analysis (TDA) \autocite{EdelsbrunnerHarer2010}, can track the birth and death of topological features (connected components, loops, voids) in the field data across different scales. The resulting "barcode" provides a unique signature for different pathological states. For example, Attractor Splintering would manifest as a proliferation of short-lived components, while the rigid structure of a Dogmatic Attractor would correspond to a single, highly persistent one.

\section{Computational Realizability Theorem}

\paragraph{Statement}
There exists a finite-dimensional discretization of Recurgent Field Theory, numerically stable and converging to the continuous solution, preserving the geometric invariants of the semantic manifold. This claim stands in dialogue with theories prposing the computability of consciousness \autocite{KochConsciousness2019}.

\paragraph{Justification}
The algorithms in this chapter demonstrate the theorem constructively. The argument rests on three pillars:
\begin{enumerate}
    \item \textbf{Standard Methods:} The algorithms employ well-understood, standard numerical methods for which stability and convergence have been proven in the literature. This includes second-order finite difference methods for partial derivatives, fourth-order Runge-Kutta integrators for ordinary differential equations, and stable matrix decomposition techniques for tensor algebra. The advanced techniques required for evolving a dynamic geometry are analogous to those developed for numerical relativity \autocite{BaumgarteShapiro2010}.
    \item \textbf{Convergence:} Consistent finite difference schemes guarantee the discretized equations converge to the continuous differential equations as the mesh resolution increases. Error estimates, such as that for the \(L^2\) norm, confirm the numerical solution approaches the true solution at a known rate.
    \item \textbf{Adaptive Techniques:} Adaptive mesh refinement in regions of high curvature and adaptive time-stepping ensure numerical stability is maintained even during the rapid evolution characteristic of pathological episodes.
\end{enumerate}
Taken together, these elements result in a computationally realizable theory admitting physically meaningful predictions. 