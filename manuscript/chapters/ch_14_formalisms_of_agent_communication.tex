\chapter{Formalisms of Agent Communication}
\label{14:formalisms_of_agent_communication}

% ------------------------------------------------------------------------------------------------

\section{Domain Structure and Cross-Domain Mapping}
\label{14.1:domain_structure_and_cross_domain_mapping}

The Semantic Manifold \(\mathcal{M}\) can be understood as a collection of partitioned submanifolds, or domains (\(\mathcal{M} = \bigcup_d \mathcal{M}_d\)), each with its own characteristic metric and organizational principles (e.g., linguistic, visual, logical domains). Hetero-recursive coupling provides the mechanism for mapping between these distinct semantic spaces. A domain translation tensor, \(T_{ij}^{(d \to d')}\), formally connects the tangent spaces of different domains, allowing coherence in one to influence another.

The recursive coupling tensor, \(R_{ijk}\), can thus be decomposed into self-referential (intra-domain) and hetero-referential (inter-domain) components:

\begin{equation}
R_{ijk}(p, q, t) = R_{ijk}^{\text{self}}(p, q, t) + R_{ijk}^{\text{hetero}}(p, q, t)
\end{equation}

where the hetero-recursive part, responsible for cross-domain mapping, is constructed from the latent recursive channel tensor \(\chi_{ijl}\) and the domain translation tensor:

\begin{equation}
R_{ijk}^{\text{hetero}}(p, q, t) = \chi_{ijl}(p, q, t) \cdot T_{lk}^{(d(q) \to d(p))}
\end{equation}

This formulation provides the fundamental mechanism for inter-agent communication and the construction of meaning across different conceptual frameworks.

% ------------------------------------------------------------------------------------------------

\section{Metaphor and Analogy as Hetero-Recursive Structures}
\label{14.2:metaphor_and_analogy_as_hetero_recursive_structures}

In this formalism, we treat metaphor as more than mere linguistic tool, but expand its description as a fundamental cognitive mechanism which structures understanding by mapping the inferential logic of a concrete source domain onto an abstract target domain. We can draw on the foundational work in conceptual metaphor theory \autocite{LakoffJohnson1980, HofstadterSander2013}. We formalize metaphors and analogies as stable, hetero-recursive mappings between a source domain \(\mathcal{S}\) and a target domain \(\mathcal{T}\). We define a metaphor as a persistent structure in the Semantic Manifold, defined by a set of high-magnitude hetero-recursive couplings:

\begin{equation}
\mathcal{M}_{\mathcal{S} \to \mathcal{T}} = \{(p, q, R_{ijk}^{\text{hetero}}(p, q, t)) \mid p \in \mathcal{S},\ q \in \mathcal{T},\ \|R_{ijk}^{\text{hetero}}(p, q, t)\| > \epsilon\}
\end{equation}

The stability of these mappings correspond to the "entrenchment" of a conceptual metaphor, which can be quantified. When such mappings form closed feedback loops, they can give rise to cross-domain amplification and conceptual blending, resulting in genuine semantic innovation. This mechanism is what allows agents to build shared understanding from differing phenomenal perspectives, forming a basis for the emergence of collective intelligence from decentralized interactions \autocite{Surowiecki2004}.

% ------------------------------------------------------------------------------------------------

\section{Inter-Agent Communication Mechanisms}
\label{14.3:inter_agent_communication_mechanisms}

Communication between agents is mediated by these mechanisms:

\textbf{Coherence Broadcast and Reception:}

\begin{equation}
C_i^{\mathrm{sent}}(p,t) = \alpha_{\mathcal{A}} \cdot \mathcal{P}_{\mathcal{A}}[C_i](p,t)
\end{equation}

\begin{equation}
C_i^{\mathrm{received}}(p,t) = \int_{\mathcal{M}} G_{\mathcal{B}}(p,q,t) \cdot C_i^{\mathrm{sent}}(q,t) \, dq
\end{equation}

where \(\mathcal{P}_{\mathcal{A}}\) is the projection operator of agent \(\mathcal{A}\) and \(G_{\mathcal{B}}\) is the reception kernel of agent \(\mathcal{B}\).

% ------------------------------------------------------------------------------------------------

\textbf{Semantic Particle Exchange:}

\begin{equation}
\mathcal{C}_{\mathcal{A}} \xrightarrow[\mathrm{geodesic\ path}]{} \mathcal{C}_{\mathcal{B}}
\end{equation}

where concept particles propagate along geodesics between agents.

% ------------------------------------------------------------------------------------------------

\textbf{Recursive Coupling Establishment:}

\begin{equation}
R_{ijk}^{\mathcal{A},\mathcal{B}}(p, q, t) = \lambda_{\mathrm{com}} \cdot \chi_{ijl}(p, q, t) \cdot T_{lk}^{(\mathcal{A} \to \mathcal{B})}
\end{equation}

representing direct recursive coupling between agent structures.

% ------------------------------------------------------------------------------------------------

\textbf{Shared Manifold Regions:}

\begin{equation}
\mathcal{S}_{\mathrm{shared}} = \mathcal{A}_{\mathrm{int}} \cap \mathcal{B}_{\mathrm{int}}
\end{equation}

defining common semantic ground.

Communication fidelity is determined by the compatibility of internal structures, metric alignment at interfaces, recursive depth, and wisdom-modulated interpretive accuracy. 